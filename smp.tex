\documentclass[minp=25, maxp=30]{cs-smp}[2013/12/23]
\usepackage{markup}

\usepackage[backend=biber,
            autocite=footnote]
            {biblatex}
\addbibresource{refs.bib}

\usepackage[hidelinks]{hyperref}
\hypersetup{
  pdfauthor={Sean Allred},
  pdftitle={SSA-Tool: A Utility for the Creation and Evaluation of Self-Stabilizing Algorithms},
  pdfsubject={Graph Theory},
  pdfkeywords={Graph,Algorithm,Distributed},
  pdfcreator={XeTeX}
}

\usepackage{makeidx}
\makeindex

\usepackage[xindy]{glossaries}

\newglossaryentry{graph}
{
  name=graph,
  description={a mathematical construct consisting of nodes and connecting edges}
}

\newglossaryentry{node}
{
  name=node,
  description={a basic component of a graph}
}

\newglossaryentry{edge}
{
  name=edge,
  description={a basic component of a graph}
}

\newglossaryentry{privilege}
{
  name=privilege,
  description={a function from a system state (a $G, v$ pair, perhaps) to a Boolean value; $\Function{G, v}{\SetBoolean}$}
}

\newglossaryentry{action}
{
  name=action,
  description={a function from state to state, perhaps $\Function{G, v}{G}$}
}

\newglossaryentry{finite state machine}
{
  name=finite state machine,
  description={uhm\todo{find a better description}}
}

\newglossaryentry{open neighborhood}
{
  name=open neighborhood,
  description={$\OpenNeighborhood{n}[G] = \Set{n^\prime}[nn^\prime \in G]$}
}

\newglossaryentry{closed neighborhood}
{
  name=closed neighboorhood,
  description={$\ClosedNeighborhood{n}[G] = \OpenNeighborhood{n}[G] \cup \Set{n}$}
}

%%% Local Variables: 
%%% mode: latex
%%% TeX-master: "smp"
%%% End: 

\makeglossaries

                                \title
                         {\texttt{ssa-tool}}%
            [A Utility for the Creation and Evaluation \\
                   of Self-Stabilizing Algorithms]

                               \author
                            {Sean Allred}%
                           [Alan Jamieson]

                                \date
                            {14 April 2013}

\begin{document}
\maketitle

\begin{abstract}
  \input{abstract.txt}
\end{abstract}

% Define what a self-stabilizing algorithm is and their applications.
% This will give an argument as to why they are necessary, why they
% are useful, and why a tool like this is warranted.
\section{Introduction}
\label{sec:introduction}
\done\todo{Author-cite Dijsktra, Goddard}
In his seminal paper on the topic~\autocite{Dijkstra:1974:SSS:361179.361202},
  \citeauthor{Dijkstra:1974:SSS:361179.361202} introduced the concept of
  self-stabilizing algorithms.
He visualized the realization of these algorithms as taking advantage of
  a very specific, extensible scenario:
\begin{displayquote}[\citeay{Dijkstra:1974:SSS:361179.361202}]
  We consider a connected graph in which
    the majority of the possible edges are missing and
    a finite state machine is placed at each node;
    machines placed in directly connected nodes are called each other's neighbors.
  For each machine one or more so-called \enquote{privileges} are defined,
    \ie boolean functions of its own state and the states of its neighbors;
    when such a boolean function is true,
    we say that the privilege is \enquote{present}.
  In order to model the undefined speed ratios of the various machines,
    we introduce a central daemon \Dash
    its replacement by a distributed daemon falls outside the scope of this article \Dash
    that can \enquote{select} one of the privileges present.
  The machine enjoying the selected privilege will then make its \enquote{move};
    \ie it is brought into a new state that is a function of
    its old state and the states of its neighbors.
  If for such a machine more than one privilege is present,
    the new state may also depend on the privilege selected.
  After completion of the move, the daemon will select a new privilege.
\end{displayquote}
This canonical definition is inherently limited, however;
  each node $n$ only knows the state of its distance-1 neighbors $v \in \ClosedNeighborhood{n}$.
\citeauthor{gairing:distance-2} introduced an approach for
  simulating distance-2 information~\autocite{gairing:distance-2},
  which was later generalized to simulating distance-$k$
  information~\autocite{goddard:ssa--k-distance} by~\citeauthor{goddard:ssa--k-distance}.
  (The approach necessarily introduced ploynomial time and space complexities.)
\citeauthor{Dijkstra:1974:SSS:361179.361202}'s restrictive-yet-practical definition
  of a self-stabilizing algorithm
  was now \emph{far} less restrictive,
  since \citeauthor{goddard:ssa--k-distance} allowed \emph{any} distance-$k$ algorithm
  to be simulated in the original definition within a reasonable complexity increase.

Since then, many such algorithms have been put forward, unsurprisingly.
\todo[cite]{find some articles}
Given their innately parallel nature,
  self-stabilizing algorithms are \emph{incredibly} useful
  for modeling and solving real-world problems.

\subsection{Related Work}
\label{sec:introduction:related-work}
In the same vein, these algorithms can be incredibly hard to test
  \Dash let alone prove \Dash
  without spending significant time implementing the algorithm
  from scratch in a programming language.
For some programming languages,
  there exist extensive graph representation libraries (\eg NetworkX~\autocite{hagberg:networkx}).
While these libraries assist in modeling the graph itself,
  they provide little to no assistance in creating, testing, or maintaining the algorithm itself.

It seems that this is partly due to the lack of definition as to
  \emph{what} exactly a self-stabilizing algorithm is.

\begin{draftvspace}{5in}
  Expand on related work.
  There are a couple of other things available (see another reference~\autocite{lafon:libcircle}, for instance),
    but I've yet to do the necessary research to say \emph{what} exactly they do.
  All I know (and cared about) was that I wasn't \emph{duplicating} functionality.
\end{draftvspace}

\subsection{Current Work}
\label{sec:introduction:current-work}
This paper presents and defends such a library and toolset.
Complete with a graphical interface which supports
  maintaining libraries of \object{Predicate} and \object{Move} objects for later assembly,
  \texttt{ssa-tool} provides researchers with tools to test algorithms in batch and by hand.

%%% Local Variables: 
%%% mode: latex
%%% TeX-master: "../smp.tex"
%%% TeX-PDF-mode: t 
%%% reftex-cite-format: "\\autocite{%l}" 
%%% End: 


% Run through the mathematical definitions of a graph, nodes/edges,
% and self-stabilizing algorithms.  In the last, stress the pairing of
% predicate and action functions.
\section{Mathematical Definitions}
\label{sec:math-defin}

The core engine must be correct above all.
To prove its correctness, all necessary definitions will be
  presented, cited as necessary, and referred to throughout the remainder of the paper.

\subsection{Graph}
\label{sec:math-defin:graphs}

A graph is a mathematical construct
  consisting of vertices $V$ and edges $E$ between them: $G = (V, E)$.
\done[don't need to worry about it]
\todo[ask]{graphs are also rep'd as sets of edges $v_1v_2$, but how are orphans notated? do I even need to worry about this?}
We define its open neighborhood of $v \in G$ to be
  the set of all nodes $u$ such that $v$ is adjacent to $u$:
  $\OpenNeighborhood{n} = \Set{v}[(n, v) \in E]$
  and the closed neighborhood is defined as $v$ together with its neighbors:
  $\ClosedNeighborhood{n} = \Union{\OpenNeighborhood{n}, \Set{n}}$.

\subsection{Self-Stabilizing Algorithm}
\label{sec:math-define:self-stab-algor}

As defined by \citeauthor{dew:sem}~\autocite{dew:sem},
  \done[Looks so much better this way]
  \todo{\texttt{citefield} the author}
  a self-stabilizing algorithm is a contruct composed of
  predicates and corresponding moves.
Both are functions of a single node and its neighbors.\footnote{%
  Note that this is not the same as the closed neighborhood of a node.
  Consider these as functions $\Function{n, \OpenNeighborhood{n}}{\Set ?}$.}
\Term{Predicates} are functions to the Boolean values~\eqref{eq:define-predicate},
  where each \Term{moves} updates the state of the system~\eqref{eq:define-move}.
When a predicate evaluates to $\BooleanTrueValue$ for any node $n$,
  it is said that $n$'s privilege is present.
\begin{align}
  \label{eq:define-predicate}
  \Function*[P]{n, \OpenNeighborhood{n}}{\SetBoolean},
  \\
  \label{eq:define-move}
  \Function*[M]{n, \OpenNeighborhood{n}}{n^\prime, \OpenNeighborhood[\prime]{n}}.
\end{align}

A self-stabilizing algorithm can thus be seen as
  a collection of these predicate--move pairs.
Formally, we may define such an algorithm $S$ to be a discrete function
  from predicates to a random move functions:
  \begin{equation}
    \label{eq:define-ssalg}
    \Function[S]{P}{\operatorname{RandomChoice}(\Set{M_1, M_2, \dots, M_N})}
  \end{equation}
\done[Nope! Not really]
\todo[ask]{Is there a better way of describing random choice?
  Any conventions that are set, notationally speaking?}
It was defined in this way to clarify the logical representation of the algorithm;
  see Section~\ref{sec:logic-repr:self-stab-algor}.
To distinguish from other uses of \Term{algorithm},
  this particular data structure will be called the \Term{rule set}.

Two properties of the algorithm must be shown
  for a self-stabilizing algorithm to be proven correct:~\autocite{arora:closure-and-convergence}
\begin{description}
\item[convergence] The algorithm must complete using a finite number of moves.
\item[closure] If the algorithm completes, the system is in a correct state.
  \done[closure and convergence are \emph{it}]
  \todo[ask]{Determine the qualities of the algorithm that must be shown.}
\end{description}

A self-stabilizing algorithm converges if and only if $\Exists{N \in \Naturals}{\ForAll{n > N, p \in P^n, v \in V}{p(v) = 0}}.$
\todo[ask]{should this be a citation or a lemma?}
\begin{comment}
  \label{eq:define-ssalg:converge}
  \Exists{N \in \Naturals}{\ForAll{n > N, p \in P^n, v \in V}{p(v) = 0}}.
\end{comment}
That is, after some finite number of moves $N \in \Naturals$,
  no predicate $p \in P$ privileges any vertex $v \in G$.
If there are no privileged nodes, no node can make a move and
  the system is considered \Term{stable}~\autocite{dew:sem}.

A self-stabilizing algorithm satisfies closure if and only
  the absence of a privileged node necessarily implies a correct overall state:
\begin{equation}
  \label{eq:define-ssalg:closure}
  \Exists{N \in \Naturals}{\ForAll{n > N, p \in P^n, v \in V}{p(v) = 0}}
  \implies \text{$G$ is in a correct state}.
\end{equation}

\paragraph{Example Algorithm and Run}
A popular problem in graph theory is \Algorithm{IndSet} \Dash
  the maximal independent set \Dash
  and it has a particularly simple self-stabilizing algorithm (see~\autoref{alg:ind-set}).
For clarity, the maximal independent set is defined as the set with the greatest cardinality
  within the power~set of the vertex~subsets $H \subseteq G$ such that
  no two nodes in $H$ are adjacent in $G$.

\begin{algorithm}[float]
  \caption{Maximal Independent Set, definition from \autocite{goddard:ssa--k-distance}}
  \label{alg:ind-set}
\begin{lstlisting}[language=ssa]
local $f$

ENTER
  if $f(i) = 0 \land \ForAll{j \in \OpenNeighborhood{i}}{f(j) = 0}$
  then:
    $f(i) = 1$

LEAVE
  if $f(i) = 1 \land \Exists{j \in \OpenNeighborhood{i}}{f(j) = 1}$
  then:
    $f(i) = 0$
\end{lstlisting}
\end{algorithm}

Considering~\autoref{alg:ind-set},
  the first rule simply says if some node $i$ is marked
  (where $f$ is a node-indexed Boolean array)
  \emph{and} no neighboring nodes $j$ are marked,
  then we \enquote*{enter} the set by setting our value in $f$ to 1.
Similarly, the second rule (\enquote*{leave})
  applys when, for some node $i$, the node is marked and a neighbor is as well.

The algorithm is thereby defined and is ready to be applied to
  some graph $G$ and nodes $i$ and neighbors $j$.
Consider the graph in~\autoref{fig:math-run:0}.
\begin{figure}
  \centering
  \includegraphics{example-image-a}
  \caption{An arbitrarily marked graph $G$ at time $t = 0$.}
  \label{fig:math-run:0}
\end{figure}
Nodes have been marked arbitrarily,
  representative of a system that is in a possibly incorrect state.
We pick a node \emph{arbitrarily}, in this case $v_0$,
  and apply the predicates from~\autoref{alg:ind-set} to the node
  to see which privileges are present.
The only present privilege is \enquote*{enter},
  so we execute the move prescribed by the rule and mark $v_0$: $f(v_0)=1$.
The graph from~\autoref{fig:math-run:0} is now~\autoref{fig:math-run:1},
  and the process begins again, perhaps now $v_4$ is picked and
  satisfies the \enquote*{leave} predicate and, with the privilege present,
  executes its moves.
\begin{figure}
  \centering
  \includegraphics{example-image-b}
  \caption{The graph from \autoref{fig:math-run:0} at time $t = 1$.}
  \label{fig:math-run:1}
\end{figure}

\paragraph{Different Models}
The paradigm given above is known as the central-daemon model~\autocite{dew:sem}.
At least one other model exists where each node acts
  of its own accord distinct from any central daemon.
\todo[cite]{Find a resource that talks about non-daemonized algorithm executions in-depth}
To simplify initial development, the daemon model was implemented
  as opposed to any truly distributed model (see~\autoref{sec:logic-repr:daemon}).
An extension may later be added, perhaps using the Python~variant Stackless~\autocite{stackless},
  that implements an alternative run-mode that uses distributed techniques.

%%% Local Variables: 
%%% mode: latex
%%% TeX-master: "../smp.tex"
%%% TeX-PDF-mode: t 
%%% reftex-cite-format: "\\autocite{%l}" 
%%% TeX-engine: xetex 
%%% TeX-command-default: "arara"
%%% End: 


% Draw from the mathematical definitions and implement them in Python.
% In doing so, create the core classes.
\section{Logical Representation}
\label{sec:logic-repr}

The entirety of this tool is written in Python~3.
The following listings serve to introduce the organization of the library and
  to serve as a reference to be used when extending this library.\footnote{%
    All development is tracked as a literate program on GitHub
    at \url{http://www.github.com/vermiculus/ssa-tool}.}

\subsection{Predicate}
\label{sec:logic-repr:predicate}

According to~\eqref{eq:define-predicate},
  a predicate is defined as some function of
  a node and its neighborhood returning a Boolean value.
This is reflected in~\autoref{lst:predicate},
  where the \object{Predicate} object maintains a pure Python function
  which is stored and called with the arguments $n$ and $\OpenNeighborhood{n}$,
  respectively.
\begin{warning}
  It is assumed that the actual neighborhood is given.
  This object is intended to be used only internally,
    and all \emph{internal} functions will indeed pass it the distance-1 neighborhood.
  It is left as an unintelligent definition to enable later extension.
\end{warning}
\lstinputlisting[
float,
caption={A predicate is a function $\Function{n, \OpenNeighborhood{n}}{\SetBoolean}$},
label={lst:predicate},
linerange=predicate-endpredicate,
]{../src/README.org}

\subsection{Move}
\label{sec:logic-repr:move}

According to~\eqref{eq:define-move},
  a move is defined as a function
  $\Function{n, \OpenNeighborhood{n}}{n^\prime, \OpenNeighborhood[\prime]{n}}$.
This is directly translated into Python as a \object{Move} object in~\autoref{lst:move}.
\begin{warning}
  As with \object{Predicate}s above,
    it is assumed that the function definition matches
    the mathematical definition given in~\eqref{eq:define-move}.
  Specifically, the function \emph{must} return a tuple of
    the new node and new neighborhood as its move.
  This object is intended to be used only internally,
    and all \emph{internal} functions will indeed pass it appropriate values.
  It is left as an unintelligent definition to enable later extension.
\end{warning}
\todo[code]{Make sure this is as advertised.}
\todo[code]{Perhaps this can be mitigated?
  Pass a simple star graph to the function to see if it returns the right thing?}
\lstinputlisting[
float,
caption={A move is a function $\Function{n, \OpenNeighborhood{n}}
  {n^\prime, \OpenNeighborhood[\prime]{n}}$},
label={lst:move},
linerange=move-endmove,
]{../src/README.org}

\subsection{Self-Stabilizing Algorithm}
\label{sec:logic-repr:self-stab-algor}

According to~\eqref{eq:define-ssalg},
  a self-stabilizing algorithm is defined as a collection
  of these \object{Predicate} and \object{Move} objects.
These must be provided as the constructor argument "ruleset",
  as in \autoref{lst:algorithm-init}.
\lstinputlisting[
float,
caption={According to~\eqref{eq:define-ssalg},
  a self-stabilizing algorithm is exactly a
  set of rules from predicate to moves.},
label={lst:algorithm-init},
linerange=algorithm-endalgorithm,
]{../src/README.org}
In an \object{Algorithm}, the "ruleset" is assumed to have
  dictionary-like structure and behavior.
This is asserted within the constructor (as can be seen in~\autoref{lst:algorithm-type-check}),
  but perhaps the clearest definition of the structure can be given as an example.
Refer to
\lstinputlisting[
float,
caption={An example structure to use as an \object{Algorithm}'s \texttt{\small ruleset}},
label={lst:algorithm-ex},
linerange={algorithm-ruleset-ex}-{end-algorithm-ruleset-ex},
]{../src/README.org}

\subsubsection{Type Checking}
\label{sec:logic-repr:self-stab-algor:type-checking}

\todo[reword]{awkward phrasing}
Since there is a strict definition on an algorithm's ruleset
  where neglecting to adhere to a specific format and behavior
  will cause other parts of the system to fail,
  a set of assertions is made upon "ruleset" that ensures at least non-crashing behavior.
These assertions (which can be reviewed in~\autoref{lst:algorithm-type-check}) check the following:
\begin{itemize}
\item "ruleset" is a collection that can be iterated upon directly,
\item all key-items in "ruleset" can be called directly,
\item all key-items in "ruleset" are functions of two arguments, and
\item all key-items in "ruleset" map to a collection of predicates, where
  \begin{itemize}
  \item each predicate can be called directly, and
  \item each predicate is a function of two arguments.
  \end{itemize}
\end{itemize}
It is quite possible that time could be saved by
  forgoing this initial type-check and simply assuming the behavior exists.
It is unknown to me whether dynamic error-catching (as in a "try"--"except" block)
  introduces a time constraint, where it would be far more detrimental to actual use.
\lstinputlisting[
float,
caption={Ensuring the rule-set we were given is usable as
  a definition of a self-stabilizing algorithm.},
label={lst:algorithm-type-check},
linerange={algorithm-ruleset-assertions}-{end-algorithm-ruleset-assertions},
]{../src/ssa/core/Algorithm.py}

\subsection{Central Daemon}
\label{sec:logic-repr:daemon}

This implementation's testing model is based of the concept of a \Term{central daemon}.
(This was the first model explicitly introduced
  by \citeauthor{dew:sem}~\autocite{dew:sem}.)
The \object{Algorithm} class plays double-duty to fulfill this role as a default behavior,
  but perhaps this should be altered.
\todo[code]{Should \object{Algorithm} actually know how to run itself,
  or should a separate \object{Daemon} object be created to handle this?
  Honestly, I'm leaning toward the latter.}
The overall implementation and interface is available in~\autoref{lst:daemon-run},
  but we will look at each part individually.
\todo[code]{The algorithm should automatically stop when it has stabilized.}
\lstinputlisting[
float=p,
caption={A generalized run of a self-stabilizing algorithm.},
label={lst:daemon-run},
linerange={daemon-run}-{end-daemon-run},
]{../src/README.org}
The daemon begins by finding assembling a list of privileged nodes.
To do this, it \emph{must} go through each node and check it for privilege.
\begin{warning}
  I consider this the greatest single bottleneck in the entire project.
  It is feasible that this can be altered into a distributed\slash threaded algorithm,
    but I haven't the knowledge to do so.
  The fact that predicates should \emph{never} write to a node or its neighbors
    allows this distribution of work among threads.
  This should be implemented in a future iteration.
\end{warning}
\lstinputlisting[
float=p,
caption={Finding privileged nodes.},
label={lst:daemon-find-privileged-nodes},
linerange={daemon-find-privileged-nodes}-{end-daemon-find-privileged-nodes},
]{../src/README.org}
Within each iteration of the loop through the nodes in the graph
  (see~\autoref{lst:daemon-find-privileged-nodes}),
  we check to see if the node is privileged or not
  (see~\autoref{lst:daemon-get-privileges}).
If it is privileged, we add the privileged node to a dictionary called "privileged_nodes".
\begin{warning}[2]
  We can also increase performance here with threading.
  Be careful of the difference, however:
    in the above we were simply reading information and
    running it through a predicate function.
  (In fact, we are doing the same here,
    but the above is a separate idea that can be sorted out.)
  The difference here is that we are altering the "privileged_nodes" data structure.
  The branch in~\autoref{lst:daemon-get-privileges} introduces a race condition:
    if two predicates $P_1, P_2$ simultaneously succeed for
    an as-yet unprivileged node $n$ at time $t$,
    time $t+1$ will result in only one unknown predicate
    $P_i$ being written to "privileged_nodes".
  Altering the data structure used for "privileged_nodes" may remove this limitation.
\end{warning}
\todo[code]{Perhaps we should just collect all predicate--move pairs in a set
  and randomly choose one?  This would be more statistically sound, but does it matter?}

\lstinputlisting[
float=p,
caption={Getting the privileges of a single node.},
label={lst:daemon-get-privileges},
linerange={daemon-get-privileges}-{end-daemon-get-privileges},
]{../src/README.org}
\lstinputlisting[
float=p,
caption={Picking a random, satisfied predicate.},
label={lst:daemon-pick-predicate},
linerange={daemon-pick-predicate}-{end-daemon-pick-predicate},
]{../src/README.org}
\lstinputlisting[
float=p,
caption={Applying a random move enabled by the satisfied predicate.},
label={lst:daemon-apply-move},
linerange={daemon-apply-move}-{end-daemon-apply-move},
]{../src/README.org}


%%% Local Variables: 
%%% mode: latex
%%% TeX-master: "../smp.tex"
%%% TeX-PDF-mode: t 
%%% reftex-cite-format: "\\autocite{%l}" 
%%% TeX-command-default: "arara"
%%% TeX-engine: xetex
%%% End: 


% Now we have the ability to logically manipulate graphs and
% algorithms.  Without an interface, however, we are still where we
% were.  Create an interface to compose an algorithm and argue the
% necessity and utility of each aspect.
\section{Interface}
\label{sec:interface-ssa}

While the core codebase has been designed and implemented
  to be easily accessible to those who wish to use it directly or extend it,
  the expected everyday interface to this project is
  the graphical tool presented below.
It fully supports the creation and automated testing of self-stabilizing algorithms.
It supports creating and documenting both predicates and moves
  using a combination of textual and graphical interfaces.
When such predicates and moves have been consolidated into a library,
  they can be assembled and organized into a self-stabilizing algorithm
  using a graphical interface.
  \todo[reword]{I feel like I'm using the word 'interface' too much}

All of these objects \Dash
  predicates, moves, and algorithms \Dash
  can be packaged into libraries that can be
  saved and distributed to colleagues.

\subsection{Predicates and Moves}
As reflected in the logical representation
  (see~\S\ref{sec:logic-repr:self-stab-algor}),
  self-stabilizing algorithms persist as a collection
  of predicates and moves.

\subsubsection{Creating}
Predicates and moves are created using a textual interface.
\todo[idea]{Could have some sort of 'raw logic' to python function processor.
  Does one exist?
}
To create one of these entities,
  use \menu{File > New > Predicate\dots}
   or \menu{File > New > Move\dots}.
\subsubsection{Documenting}
\subsubsection{Testing}
\subsubsection{Maintaining}
\subsubsection{Distributing}

\subsection{Algorithms}

\subsubsection{Assembly}
\subsubsection{Documenting}
\subsubsection{Testing}
\subsubsection{Maintaining}
\subsubsection{Distributing}


%%% Local Variables: 
%%% mode: latex
%%% TeX-master: "../smp.tex"
%%% reftex-cite-format: "\\autocite{%l}" 
%%% TeX-PDF-mode: t 
%%% End: 


% We have the ability to create algorithms, but nothing to test them
% on.  Introduce an interface to create graphs and import/export them
% to common formats.  This section is simply presentation; it needs no
% argument.
\section{Graph Explorer}
\label{sec:graph-builder}

\subsection{From Scratch}
\subsection{Interoperability}
Supported by NetworkX~\parencite{hagberg:networkx}
\subsubsection{GraphML}
\paragraph{Importing}
\paragraph{Exporting}

\subsubsection{GEXF}
\paragraph{Importing}
\paragraph{Exporting}

%%% Local Variables: 
%%% mode: latex
%%% TeX-master: "../smp.tex"
%%% reftex-cite-format: "\\autocite{%l}" 
%%% TeX-PDF-mode: t 
%%% TeX-command-default: "arara"
%%% End: 


% We have algorithms and graphs to test them on, but nothing to
% facilitate the testing.  Introduce an interface to load algorithms
% and test with specific graphs (or those of an arbitrary degree).
\section{Testing Interface}
\label{sec:interface-testing}

\subsection{Test Graphs}
\subsubsection{Creating from Scratch}
\subsubsection{Loading from Files}
\label{sec:interface-testing:import}
Supported by NetworkX~\autocite{hagberg:networkx}
\begin{figure}
  \centering
  \includegraphics{example-image-a}
  \caption{Importing an algorithm test (packaged as algorithm + graph)}
  \label{fig:iface:alg-test-import}
\end{figure}
\begin{figure}
  \centering
  \includegraphics{example-image-b}
  \caption{Exporting an algorithm test (packaged as algorithm + text)}
  \label{fig:iface:alg-test-export}
\end{figure}
\paragraph{GraphML}
\subparagraph{Importing}
\subparagraph{Exporting}

\paragraph{GEXF}
\subparagraph{Importing}
\subparagraph{Exporting}

\subsection{Loading Algorithms} % from a file
\todo{Is this section obsolete by \S\ref{sec:interface-testing-import}?}
\subsection{Applying Algorithms}
\subsubsection{Controlled Iteration}
\begin{figure}
  \centering
  \includegraphics{example-image-a}
  \caption{Graphical Play\slash Pause\slash FF\slash Rewind\slash\dots}
  \label{fig:iface:alg-test-controlled}
\end{figure}
\begin{draftvspace}{2in}
  Controlled iteration.
\end{draftvspace}
\subsubsection{Uncontrolled Iteration}
\begin{figure}
  \centering
  \includegraphics{example-image-b}
  \caption{Batch Mode Run}
  \label{fig:iface:alg-test-batch}
\end{figure}
\begin{draftvspace}{2in}
  Batch mode.
\end{draftvspace}
\subsubsection{Keeping History}
\begin{draftvspace}{1in}
  Saveable as an animated GEXF.
\end{draftvspace}

%%% Local Variables: 
%%% mode: latex
%%% TeX-master: "../smp.tex"
%%% reftex-cite-format: "\\autocite{%l}" 
%%% TeX-PDF-mode: t
%%% TeX-command-default: "arara"
%%% End: 


\printglossaries

\printbibliography
\appendix

%%%%%%%%%%%%%%%%%%%%%%%%%%%%%%%%%%%%%%%%%%%%%%%%%%%%%%%%%%%%%%%%
%%% Sandbox %%%%%%%%%%%%%%%%%%%%%%%%%%%%%%%%%%%%%%%%%%%%%%%%%%%%
%%%%%%%%%%%%%%%%%%%%%%%%%%%%%%%%%%%%%%%%%%%%%%%%%%%%%%%%%%%%%%%%

\section{sandbox}
A sample citation: \autocites[26.1--26.45]{atallah2009algorithms}{Chen1991147}.

\end{document}

%%% Local Variables: 
%%% mode: latex
%%% TeX-master: t
%%% TeX-PDF-mode: t 
%%% truncate-lines: nil 
%%% End: 
