% arara: xelatex
% arara: biber
% arara: makeglossaries
% arara: xelatex
% arara: xelatex
% arara: clean: { files: [ smp.aux, smp.log, smp.out ] }
% arara: clean: { files: [ smp.toc, smp.lot, smp.lof ] }
% arara: clean: { files: [ smp.bbl, smp.bcf, smp.blg, smp.run.xml ] }
% arara: clean: { files: [ smp.glg, smp.glo, smp.gls ] }
% arara: clean: { files: [ smp.idx, smp.xdy ] }
\documentclass[
minp=25,
maxp=30,
letterpaper,
]{cs-smp}[2013/12/23]

\usepackage{amsmath, amssymb, amsthm}
\usepackage{smcm-math, smcm-danda}
\usepackage{markup}
\usepackage{csquotes}
\usepackage{listings}
\lstset{
  rangeprefix=\#\%\ ,
  rangesuffix=\ \%\#,
  includerangemarker=false,
  language=Python,
}
\usepackage[marginpar]{todo}

\usepackage{microtype}

\usepackage[
backend=biber,
autocite=footnote,
style=authoryear,
]{biblatex}

\addbibresource{refs.bib}

\usepackage{tikz}
\usetikzlibrary{automata}

\usepackage[hidelinks]{hyperref}
\hypersetup{
  pdfauthor={Sean Allred},
  pdftitle={SSA-Tool: A Utility for the Creation and Evaluation of Self-Stabilizing Algorithms},
  pdfsubject={Graph Theory},
  pdfkeywords={Graph,Algorithm,Distributed},
  pdfcreator={XeTeX}
}

\usepackage{menukeys}           % actually has to be loaded after hyperref?
\usepackage{makeidx}
\makeindex

\usepackage[
  xindy,                        % use xindy for glossary index compilation
]{glossaries}

\newglossaryentry{graph}
{
  name=graph,
  description={a mathematical construct consisting of nodes and connecting edges}
}

\newglossaryentry{node}
{
  name=node,
  description={a basic component of a graph}
}

\newglossaryentry{edge}
{
  name=edge,
  description={a basic component of a graph}
}

\newglossaryentry{predicate}
{
  name=predicate,
  description={some function $\Function{n, \OpenNeighborhood{n}}{\SetBoolean}$}
}

\newglossaryentry{present}
{
  name=present,
  description={when referring to privileges, a node $n$'s privilege is present when some predicate $P(n) = \BooleanTrueValue$}
}

\newglossaryentry{privilege}
{
  name=privilege,
  description={a state-based quality of a node that gives it the right to possibly make a move}
}

\newglossaryentry{move}
{
  name=move,
  description={some function $\Function{G, v}{G^\prime}$}
}

\newacronym[
  description={uhm\todo{find a better description}}
]{fsm}{\textsc{fsm}}{finite state machine}

\newacronym[
  longplural={non-deterministic finite automata},
  description={a model of computation\todo{expand this in the glossary, but how?}}
]{nfa}{\textsc{nfa}}{non-deterministic finite automoton}

\newacronym[
  longplural={deterministic finite automata},
  description={a model of computation\todo{expand this in the glossary, but how?}}
]{dfa}{\textsc{dfa}}{deterministic finite automoton}

\newglossaryentry{open neighborhood}
{
  name=open neighborhood,
  description={$\OpenNeighborhood{n}[G] = \Set{n^\prime}[nn^\prime \in G]$}
}

\newglossaryentry{closed neighborhood}
{
  name=closed neighboorhood,
  description={$\ClosedNeighborhood{n}[G] = \OpenNeighborhood{n}[G] \cup \Set{n}$}
}

%%% Local Variables: 
%%% mode: latex
%%% TeX-master: "smp"
%%% TeX-PDF-mode: t 
%%% End: 

\makeglossaries

                                \title
                         {\texttt{ssa-tool}}%
            [A Utility for the Creation and Evaluation \\
                   of Self-Stabilizing Algorithms]

                               \author
                            {Sean Allred}%
                           [Alan Jamieson]

                                \date
                            {14 April 2013}

\begin{document}
\maketitle

\begin{abstract}
  \input{abstract.txt}
\end{abstract}

\tableofcontents
\listoffigures
\listoftables

\clearpage
\pagestyle{plain}

% Define what a self-stabilizing algorithm is and their applications.
% This will give an argument as to why they are necessary, why they
% are useful, and why a tool like this is warranted.
\section{Introduction}
\label{sec:introduction}
\done\todo{Author-cite Dijsktra, Goddard}
In his seminal paper on the topic~\autocite{Dijkstra:1974:SSS:361179.361202},
  \citeauthor{Dijkstra:1974:SSS:361179.361202} introduced the concept of
  self-stabilizing algorithms.
He visualized the realization of these algorithms as taking advantage of
  a very specific, extensible scenario:
\begin{displayquote}[\citeay{Dijkstra:1974:SSS:361179.361202}]
  We consider a connected graph in which
    the majority of the possible edges are missing and
    a finite state machine is placed at each node;
    machines placed in directly connected nodes are called each other's neighbors.
  For each machine one or more so-called \enquote{privileges} are defined,
    \ie boolean functions of its own state and the states of its neighbors;
    when such a boolean function is true,
    we say that the privilege is \enquote{present}.
  In order to model the undefined speed ratios of the various machines,
    we introduce a central daemon \Dash
    its replacement by a distributed daemon falls outside the scope of this article \Dash
    that can \enquote{select} one of the privileges present.
  The machine enjoying the selected privilege will then make its \enquote{move};
    \ie it is brought into a new state that is a function of
    its old state and the states of its neighbors.
  If for such a machine more than one privilege is present,
    the new state may also depend on the privilege selected.
  After completion of the move, the daemon will select a new privilege.
\end{displayquote}
This canonical definition is inherently limited, however;
  each node $n$ only knows the state of its distance-1 neighbors $v \in \ClosedNeighborhood{n}$.
\citeauthor{gairing:distance-2} introduced an approach for
  simulating distance-2 information~\autocite{gairing:distance-2},
  which was later generalized to simulating distance-$k$
  information~\autocite{goddard:ssa--k-distance} by~\citeauthor{goddard:ssa--k-distance}.
  (The approach necessarily introduced ploynomial time and space complexities.)
\citeauthor{Dijkstra:1974:SSS:361179.361202}'s restrictive-yet-practical definition
  of a self-stabilizing algorithm
  was now \emph{far} less restrictive,
  since \citeauthor{goddard:ssa--k-distance} allowed \emph{any} distance-$k$ algorithm
  to be simulated in the original definition within a reasonable complexity increase.

Since then, many such algorithms have been put forward, unsurprisingly.
\todo[cite]{find some articles}
Given their innately parallel nature,
  self-stabilizing algorithms are \emph{incredibly} useful
  for modeling and solving real-world problems.

\subsection{Related Work}
\label{sec:introduction:related-work}
In the same vein, these algorithms can be incredibly hard to test
  \Dash let alone prove \Dash
  without spending significant time implementing the algorithm
  from scratch in a programming language.
For some programming languages,
  there exist extensive graph representation libraries (\eg NetworkX~\autocite{hagberg:networkx}).
While these libraries assist in modeling the graph itself,
  they provide little to no assistance in creating, testing, or maintaining the algorithm itself.

It seems that this is partly due to the lack of definition as to
  \emph{what} exactly a self-stabilizing algorithm is.

\begin{draftvspace}{5in}
  Expand on related work.
  There are a couple of other things available (see another reference~\autocite{lafon:libcircle}, for instance),
    but I've yet to do the necessary research to say \emph{what} exactly they do.
  All I know (and cared about) was that I wasn't \emph{duplicating} functionality.
\end{draftvspace}

\subsection{Current Work}
\label{sec:introduction:current-work}
This paper presents and defends such a library and toolset.
Complete with a graphical interface which supports
  maintaining libraries of \object{Predicate} and \object{Move} objects for later assembly,
  \texttt{ssa-tool} provides researchers with tools to test algorithms in batch and by hand.
\begin{draftvspace}{4in}
  Absolutely \emph{must} expand on this more.
  Go into what exactly happens and why it is necessary.
\end{draftvspace}

%%% Local Variables: 
%%% mode: latex
%%% TeX-master: "../smp.tex"
%%% TeX-PDF-mode: t 
%%% reftex-cite-format: "\\autocite{%l}" 
%%% TeX-command-default: "arara"
%%% End: 


% Run through the mathematical definitions of a graph, nodes/edges,
% and self-stabilizing algorithms.  In the last, stress the pairing of
% predicate and action functions.
\section{Mathematical Definitions}
\label{sec:math-defin}

The core engine must be correct above all.
To prove its correctness, all necessary definitions will be
  presented, cited as necessary, and referred to throughout the remainder of the paper.

\subsection{Graph}
\label{sec:math-defin:graphs}

A graph is a mathematical construct
  consisting of vertices $V$ and edges $E$ between them: $G = (V, E)$.
\done[don't need to worry about it]
\todo[ask]{graphs are also rep'd as sets of edges $v_1v_2$, but how are orphans notated? do I even need to worry about this?}
We define its open neighborhood of $v \in G$ to be
  the set of all nodes $u$ such that $v$ is adjacent to $u$:
  $\OpenNeighborhood{n} = \Set{v}[(n, v) \in E]$
  and the closed neighborhood is defined as $v$ together with its neighbors:
  $\ClosedNeighborhood{n} = \Union{\OpenNeighborhood{n}, \Set{n}}$.

\subsection{Self-Stabilizing Algorithm}
\label{sec:math-define:self-stab-algor}

As defined by \citeauthor{dew:sem}~\autocite{dew:sem},
  \done[Looks so much better this way]
  \todo{\texttt{citefield} the author}
  a self-stabilizing algorithm is a contruct composed of
  predicates and corresponding moves.
Both are functions of a single node and its neighbors.\footnote{%
  Note that this is not the same as the closed neighborhood of a node.
  Consider these as functions $\Function{n, \OpenNeighborhood{n}}{\Set ?}$.}
\Term{Predicates} are functions to the Boolean values~\eqref{eq:define-predicate},
  where each \Term{moves} updates the state of the system~\eqref{eq:define-move}.
When a predicate evaluates to $\BooleanTrueValue$ for any node $n$,
  it is said that $n$'s privilege is present.
\begin{align}
  \label{eq:define-predicate}
  \Function*[P]{n, \OpenNeighborhood{n}}{\SetBoolean},
  \\
  \label{eq:define-move}
  \Function*[M]{n, \OpenNeighborhood{n}}{n^\prime, \OpenNeighborhood[\prime]{n}}.
\end{align}

A self-stabilizing algorithm can thus be seen as
  a collection of these predicate--move pairs.
Formally, we may define such an algorithm $S$ to be a discrete function
  from predicates to a random move functions:
  \begin{equation}
    \label{eq:define-ssalg}
    \Function[S]{P}{\operatorname{RandomChoice}(\Set{M_1, M_2, \dots, M_N})}
  \end{equation}
\done[Nope! Not really]
\todo[ask]{Is there a better way of describing random choice?
  Any conventions that are set, notationally speaking?}
It was defined in this way to clarify the logical representation of the algorithm;
  see Section~\ref{sec:logic-repr:self-stab-algor}.
To distinguish from other uses of \Term{algorithm},
  this particular data structure will be called the \Term{rule set}.

Two properties of the algorithm must be shown
  for a self-stabilizing algorithm to be proven correct:~\autocite{arora:closure-and-convergence}
\begin{description}
\item[convergence] The algorithm must complete using a finite number of moves.
\item[closure] If the algorithm completes, the system is in a correct state.
  \done[closure and convergence are \emph{it}]
  \todo[ask]{Determine the qualities of the algorithm that must be shown.}
\end{description}

A self-stabilizing algorithm converges if and only if $\Exists{N \in \Naturals}{\ForAll{n > N, p \in P^n, v \in V}{p(v) = 0}}.$
\todo[ask]{should this be a citation or a lemma?}
\begin{comment}
  \label{eq:define-ssalg:converge}
  \Exists{N \in \Naturals}{\ForAll{n > N, p \in P^n, v \in V}{p(v) = 0}}.
\end{comment}
That is, after some finite number of moves $N \in \Naturals$,
  no predicate $p \in P$ privileges any vertex $v \in G$.
If there are no privileged nodes, no node can make a move and
  the system is considered \Term{stable}~\autocite{dew:sem}.

A self-stabilizing algorithm satisfies closure if and only
  the absence of a privileged node necessarily implies a correct overall state:
\begin{equation}
  \label{eq:define-ssalg:closure}
  \Exists{N \in \Naturals}{\ForAll{n > N, p \in P^n, v \in V}{p(v) = 0}}
  \implies \text{$G$ is in a correct state}.
\end{equation}

\paragraph{Example Algorithm and Run}
\tikzset{
  marked/.style={
    fill=blue!15
  }
}
A popular problem in graph theory is \Algorithm{IndSet} \Dash
  the maximal independent set \Dash
  and it has a particularly simple self-stabilizing algorithm (see~\autoref{alg:ind-set}).
For clarity, the maximal independent set is defined as the set with the greatest cardinality
  within the power~set of the vertex~subsets $H \subseteq G$ such that
  no two nodes in $H$ are adjacent in $G$.

\begin{algorithm}[float]
  \caption{Maximal Independent Set, definition from \autocite{goddard:ssa--k-distance}}
  \label{alg:ind-set}
\begin{lstlisting}[language=ssa]
local $f$

ENTER
  if $f(i) = 0 \land \ForAll{j \in \OpenNeighborhood{i}}{f(j) = 0}$
  then:
    $f(i) = 1$

LEAVE
  if $f(i) = 1 \land \Exists{j \in \OpenNeighborhood{i}}{f(j) = 1}$
  then:
    $f(i) = 0$
\end{lstlisting}
\end{algorithm}

Considering~\autoref{alg:ind-set},
  the first rule simply says if some node $i$ is marked
  (where $f$ is a node-indexed Boolean array)
  \emph{and} no neighboring nodes $j$ are marked,
  then we \enquote*{enter} the set by setting our value in $f$ to 1.
Similarly, the second rule (\enquote*{leave})
  applys when, for some node $i$, the node is marked and a neighbor is as well.

The algorithm is thereby defined and is ready to be applied to
  some graph $G$ and nodes $i$ and neighbors $j$.
Consider the graph in~\autoref{fig:math-run:0}.
Nodes have been marked arbitrarily,
  representative of a system that is in a possibly incorrect state.
We pick a node \emph{arbitrarily}, in this case $v_5$,
  and apply the predicates from~\autoref{alg:ind-set} to the node
  to see which privileges are present.
The only present privilege is \enquote*{enter},
  so we execute the move prescribed by the rule and mark $v_5$: $f(v_5)=1$.
The graph from~\autoref{fig:math-run:0} is now~\autoref{fig:math-run:1},
  and the process begins again, perhaps now $v_4$ is picked and
  satisfies the \enquote*{leave} predicate and, with the privilege present,
  executes its moves.
\begin{figure}[p]
  \centering
  \begin{tikzpicture}
      \node[graph vertex, marked] (1) at (0 , 3) {$v_1$};
      \node[graph vertex,       ] (2) at (0 , 6) {$v_2$};
      \node[graph vertex,       ] (3) at (3 , 3) {$v_3$};
      \node[graph vertex, marked] (4) at (3 , 6) {$v_4$};
      \node[graph vertex,       ] (5) at (5 , 8) {$v_5$};

      \graph {
        (1) -- (3) -- (4) -- (2) --[bend left] (5),
        (1) -- (2),
        (1) -- (4),
        (3) --[bend right] (5)
      };
  \end{tikzpicture}
  \caption{An arbitrarily marked graph $G$ at time $t = 0$.}
  \label{fig:math-run:0}
\end{figure}
\begin{figure}[p]
  \centering
  \begin{tikzpicture}
      \node[graph vertex, marked] (1) at (0 , 3) {$v_1$};
      \node[graph vertex,       ] (2) at (0 , 6) {$v_2$};
      \node[graph vertex,       ] (3) at (3 , 3) {$v_3$};
      \node[graph vertex, marked] (4) at (3 , 6) {$v_4$};
      \node[graph vertex, marked] (5) at (5 , 8) {$v_5$};

      \graph {
        (1) -- (3) -- (4) -- (2) --[bend left] (5),
        (1) -- (2),
        (1) -- (4),
        (3) --[bend right] (5)
      };
  \end{tikzpicture}
  \caption{The graph from \autoref{fig:math-run:0} at time $t = 1$.}
  \label{fig:math-run:1}
\end{figure}
\begin{figure}[p]
  \centering
  \begin{tikzpicture}
      \node[graph vertex, marked] (1) at (0 , 3) {$v_1$};
      \node[graph vertex,       ] (2) at (0 , 6) {$v_2$};
      \node[graph vertex,       ] (3) at (3 , 3) {$v_3$};
      \node[graph vertex,       ] (4) at (3 , 6) {$v_4$};
      \node[graph vertex, marked] (5) at (5 , 8) {$v_5$};

      \graph {
        (1) -- (3) -- (4) -- (2) --[bend left] (5),
        (1) -- (2),
        (1) -- (4),
        (3) --[bend right] (5)
      };
  \end{tikzpicture}
  \caption{The graph from \autoref{fig:math-run:0} at time $t = 2$.  No more privileges are present for any node $v_i$, so $G$ is now called \Term{stable} and the marked nodes represent a maximal independant set.}
  \label{fig:math-run:1}
\end{figure}

\paragraph{Different Models}
The paradigm given above is known as the central-daemon model~\autocite{dew:sem}.
At least one other model exists where each node acts
  of its own accord distinct from any central daemon.
\todo[cite]{Find a resource that talks about non-daemonized algorithm executions in-depth}
To simplify initial development, the daemon model was implemented
  as opposed to any truly distributed model (see~\autoref{sec:logic-repr:daemon}).
An extension may later be added, perhaps using the Python~variant Stackless~\autocite{stackless},
  that implements an alternative run-mode that uses distributed techniques.

%%% Local Variables: 
%%% mode: latex
%%% TeX-master: "../smp.tex"
%%% TeX-PDF-mode: t 
%%% reftex-cite-format: "\\autocite{%l}" 
%%% TeX-engine: xetex 
%%% TeX-command-default: "arara"
%%% End: 


% Draw from the mathematical definitions and implement them in Python.
% In doing so, create the core classes.

\section{Logical Representation}
\label{sec:logic-repr}

The entirety of this tool is written in Python~3.

%%% Local Variables: 
%%% mode: latex
%%% TeX-master: "../smp.tex"
%%% TeX-PDF-mode: t 
%%% reftex-cite-format: "\\autocite{%l}" 
%%% End: 


% Now we have the ability to logically manipulate graphs and
% algorithms.  Without an interface, however, we are still where we
% were.  Create an interface to compose an algorithm and argue the
% necessity and utility of each aspect.
\section{Interface}
\label{sec:interface-ssa}

While the core codebase has been designed and implemented
  to be easily accessible to those who wish to use it directly or extend it,
  the expected everyday interface to this project is
  the graphical tool presented below.
It fully supports the creation and automated testing of self-stabilizing algorithms.
It supports creating and documenting both predicates and moves
  using a combination of textual and graphical interfaces.
When such predicates and moves have been consolidated into a library,
  they can be assembled and organized into a self-stabilizing algorithm
  using a graphical interface.
  \todo[reword]{I feel like I'm using the word 'interface' too much}

All of these objects \Dash
  predicates, moves, and algorithms \Dash
  can be packaged into libraries that can be
  saved and distributed to colleagues.

\subsection{Predicates and Moves}
As reflected in the logical representation
  (see~\S\ref{sec:logic-repr:self-stab-algor}),
  self-stabilizing algorithms persist as a collection
  of predicates and moves.

\subsubsection{Creating}
Predicates and moves are created using a textual interface.
\todo[idea]{Could have some sort of 'raw logic' to python function processor.
  Does one exist?}
To create one of these entities,
  use \menu{File > New > Predicate\dots}
   or \menu{File > New > Move\dots}.
This presents you with a screen similar to~\autoref{fig:iface:create-pred},
  allowing you to provide the code (in Python\autocite{python3:ref}).
A template is provided for you to begin working from,
  along with short examples of how to access nodes, edges, node data, and edge data.
(If you are familiar with the toole,
  the syntax is from NetworkX~\autocite{hagberg:networkx}.)
\begin{figure}
  \centering
  \includegraphics{example-image-a}
  \caption{When creating an entirely new predicate or move,
    you are presented with a helpful template to start working from.}
  \label{fig:iface:create-pred}
\end{figure}

\subsubsection{Documenting}
\subsubsection{Testing}
\subsubsection{Maintaining}
\subsubsection{Distributing}

\subsection{Algorithms}

\subsubsection{Assembly}
\subsubsection{Documenting}
\subsubsection{Testing}
\subsubsection{Maintaining}
\subsubsection{Distributing}


%%% Local Variables: 
%%% mode: latex
%%% TeX-master: "../smp.tex"
%%% reftex-cite-format: "\\autocite{%l}" 
%%% TeX-PDF-mode: t 
%%% TeX-command-default: "arara"
%%% End: 


% We have the ability to create algorithms, but nothing to test them
% on.  Introduce an interface to create graphs and import/export them
% to common formats.  This section is simply presentation; it needs no
% argument.
%
\section{Graph Builder}
\label{sec:graph-builder}


%%% Local Variables: 
%%% mode: latex
%%% TeX-master: "../smp.tex"
%%% reftex-cite-format: "\\autocite{%l}" 
%%% End: 


% We have algorithms and graphs to test them on, but nothing to
% facilitate the testing.  Introduce an interface to load algorithms
% and test with specific graphs (or those of an arbitrary degree).
\section{Testing Interface}
\label{sec:interface-testing}

Testing is currently only supported via the text interface (see~\autoref{task:testing-interface}).
See~\autoref{lst:test-cmd} for an example run of \Algorithm{IndSet} on a star graph.
Valid graph input formats are \fileformat{gexf}, \fileformat{gml}, and \fileformat{yaml}.

\begin{lstlisting}[language={},float=!h,caption={A model testing session using the command-line utility.},label={lst:test-cmd}]
$ ./ssa-tool run "Independent Set" \
             from examples/ind-set.ssax \
             on test.gml \
             --non-interactive
Welcome to SSA-Tool, version 1.
Running:
  Algorithm: "Independent Set"
       from: "examples/ind-set.ssax"
         on: "test.gml"
    (format: "gml")
Read Graph:
[(0, {'id': 0, 'label': 0, 'marked': 0}),
 (1, {'id': 1, 'label': 1, 'marked': 1}),
 (2, {'id': 2, 'label': 2, 'marked': 0}),
 (3, {'id': 3, 'label': 3, 'marked': 0}),
 (4, {'id': 4, 'label': 4, 'marked': 0})]
History:
[{'move': move 'mark node',
  'neighbors': {0: {'id': 0, 'label': 0, 'marked': 0}},
  'new neighbors': {0: {'id': 0, 'label': 0, 'marked': 0}},
  'new node': 2,
  'node': (2, {'id': 2, 'label': 2, 'marked': 0})},
 {'move': move 'mark node',
  'neighbors': {0: {'id': 0, 'label': 0, 'marked': 0}},
  'new neighbors': {0: {'id': 0, 'label': 0, 'marked': 0}},
  'new node': 3,
  'node': (3, {'id': 3, 'label': 3, 'marked': 0})},
 {'move': move 'mark node',
  'neighbors': {0: {'id': 0, 'label': 0, 'marked': 0}},
  'new neighbors': {0: {'id': 0, 'label': 0, 'marked': 0}},
  'new node': 4,
  'node': (4, {'id': 4, 'label': 4, 'marked': 0})}]
Stable Graph:
[(0, {'id': 0, 'label': 0, 'marked': 0}),
 (1, {'id': 1, 'label': 1, 'marked': 1}),
 (2, {'id': 2, 'label': 2, 'marked': True}),
 (3, {'id': 3, 'label': 3, 'marked': True}),
 (4, {'id': 4, 'label': 4, 'marked': True})]
\end{lstlisting}

%%% Local Variables: 
%%% mode: latex
%%% TeX-master: "../smp.tex"
%%% reftex-cite-format: "\\autocite{%l}" 
%%% TeX-PDF-mode: t
%%% TeX-command-default: "arara"
%%% TeX-engine: xetex
%%% End: 


% Dunno
\section{Reflection}
\label{sec:reflection}

\begin{itemize}
\item The hardest part was definitely finding definitions.
  It would be good if there were a definitive resource
  (or even general agreement) on these things.
\item 
\end{itemize}

%%% Local Variables: 
%%% mode: latex
%%% TeX-master: "../smp.tex"
%%% TeX-PDF-mode: t 
%%% reftex-cite-format: "\\autocite{%l}" 
%%% End: 


\appendix

\newpage
\printglossaries

\newpage
\printbibliography

\todos
\end{document}

%%% Local Variables: 
%%% mode: latex
%%% TeX-master: t
%%% TeX-PDF-mode: t 
%%% truncate-lines: nil 
%%% TeX-engine: xetex 
%%% End: 
