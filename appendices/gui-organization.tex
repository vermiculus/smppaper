\section{Logical Organization of Graphical Interface}
\label{sec:gui-org}

The graphical interface is logically organized into several dictionaries based on tabs
  (see~\autoref{tab:gui-variables}).
The keys of these dictionaries are string-descriptions of the purpose of the entity,
  and the value is either a single variable or function (in the cases of "??v" and "??f")
  or a pair (as a "tuple") of positional coordinates and the widget itself (within "??w").
The coordinates are relative to the upper-left corner of the containing frame.
If the first element of the tuple is "None" (instead of a pair),
  the widget will simply be \enquote*{packed} instead of \enquote*{placed}.
To create the widgets, a utility function is declared that will assemble the widget
  and log any information about it that is necessary,
  providing a centralized interface to manage data bindings (see~\autoref{lst:gui:wgt-creator}).

\begin{lstlisting}[float,caption={A standardized widget creator},label={lst:gui:wgt-creator}]
def new(cls, widget_dictionary, name, **kwargs):
    print('Creating widget {0:<14} under {1}'.format(cls.__name__, name))
    return cls(widget_dictionary[name][1], **kwargs)
\end{lstlisting}
\begin{table}
  \centering
  \begin{tabular}{lcccc}
    \toprule
    Type      & File Manager & Algorithm & Predicate & Move  \\ \midrule
    Variables & "fmv"        & "agv"     & "pdv"     & "mvv" \\
    Functions & "fmf"        & "agf"     & "pdf"     & "mvf" \\
    Widgets   & "fmw"        & "agw"     & "pdw"     & "mvw" \\
    \bottomrule
  \end{tabular}
  \caption{Standardized interface dictionaries, indexed by description.}
  \label{tab:gui-variables}
\end{table}

For example, to add some text entry field for a predicate's name,
  I would call "new" from~\autoref{lst:gui:wgt-creator} with the arguments
  "(Entry, pdw, 'tab', textvariable = pdv['predicate name])".

Bindings between graphical elements and their logical counterparts
  are maintained in a dictionary called "bind", primarily indexed by class
  and secondarily indexed by the object's name.
This dictionary is reset upon calling "refresh",
  which purges data from the interface and rebuilds it using
  the "bundle" data structure.

\paragraph{About the Dictionary-Based Approach}
The choice to use dictionaries was spurred by
  what has proven to be a common problem in my own software implementations.
Within the dynamic functions of the interface
  (\eg the function to transfer moves to and from a rule),
  it is always possible that the widget the function is altering has not yet been created.
By using dictionaries, resolution is deferred until the interface is completely built,
  thereby side-stepping the need to carefully order the creation of widgets
  and instead order them in a way that makes sense or is aesthetically pleasing.

%%% Local Variables:
%%% mode: latex
%%% TeX-master: "../smp.tex"
%%% TeX-engine: xetex
%%% TeX-PDF-mode: t
%%% End:
