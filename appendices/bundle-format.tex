\section{Bundle Description Document Specification}
\label{sec:bundle-descr-docum}

\begin{figure}[p]
  \centering
  \small
  \begin{minipage}{.4\textwidth}
    \dirtree{%
      .1 ind-set.ssax.
      .2 bundle.yaml.
      .2 moves.
      .3 mark.py.
      .3 unmark.py.
      .2 predicates.
      .3 should-mark.py.
      .3 should-unmark.py.
    }
  \end{minipage}
  \caption{The layout of an example algorithm, \Algorithm{IndSet}}
  \label{fig:interop:ssax-structure}
\end{figure}
\begin{lstlisting}[float=p,caption={\directory{moves/mark.py}},label={lst:ex:mark}]
v['marked'] = True
\end{lstlisting}
\begin{lstlisting}[float=p,caption={\directory{moves/unmark.py}},label={lst:ex:unmark}]
v['marked'] = False
\end{lstlisting}
\begin{lstlisting}[float=p,caption={\directory{predicates/marked-and-neighbor-marked.py}},label={lst:ex:should-unmark}]
marked = v['marked']
neighbor_marked = any(map(lambda n: n['marked'], N))
return marked and neighbor_marked
\end{lstlisting}
\begin{lstlisting}[float=p,caption={\directory{predicates/unmarked-and-neighbors-unmarked.py}},label={lst:ex:should-mark}]
marked = v['marked']
neighbor_marked = any(map(lambda n: n['marked'], N))
return not (marked or neighbor_marked)
\end{lstlisting}
\begin{lstlisting}[float=p, language=yaml, basicstyle=\ttfamily\footnotesize, caption={\directory{bundle.yaml}}, label={lst:interop:yaml-alg}]
--- !Move
name: mark node
description: mark this node
author: Sean Allred
date: 2014-05-17
tex: '"marked"(n) = 1'
filename: mark.py
--- !Move
name: unmark node
description: unmark this node
author: Sean Allred
date: 2014-05-17
tex: '"marked"(n) = 0'
filename: unmark.py
--- !Predicate
name: marked and neighbor marked
filename: marked-and-neighbor-marked.py
description: two adjacent nodes are marked
author: Sean Allred
date: 2014-05-17
tex: '"marked"(n) = 1 \land \exists v \in N(n) : "marked"(v) = 1'
--- !Predicate
name: unmarked and neighbors unmarked
filename: unmarked-and-neighbors-unmarked.py
description: All nodes in N[n] are unmarked
author: Sean Allred
date: 2014-05-17
tex: '"marked"(n) = 0 \land \forall v \in N(n) : "marked"(v) = 0'
--- !Algorithm
name: Independent Set
author: Sean Allred
date: 2014-05-17
rules:
  - !Rule
    name: unmark if touching
    description: if a neighbor is marked, unmark
    author: Sean Allred
    date: 2014-05-17
    predicate: marked and neighbor marked
    moves: [ unmark node ]
  - !Rule
    name: mark if able
    description: if no neighbors are marked, mark
    author: Sean Allred
    date: 2014-05-17
    predicate: unmarked and neighbors unmarked
    moves: [ mark node ]
\end{lstlisting}

The file format that "ssa-tool" natively works with
  is actually an OS\,X-style \enquote*{package} \Dash 
  it is a standardized directory with a specific extension.
The recommended extension to use is \fileformat{ssax},
  but the tool itself does not require it.
An example directory structure can be seen in~\autoref{fig:interop:ssax-structure},
  but the following sections will formally describe each portion.

\subsection{Predicates and Moves}
\label{sec:bundle:predicates-moves}

Predicates and moves are stored under the
  separate directories "predicates" and "moves" as in~\autoref{fig:interop:ssax-structure}.
Each file in these directories is a
  (potentially very short, see~\autoref{lst:logic:pred-def}) Python~3 script
  that defines the body of a templated function
  which is hardcoded as in to~\autoref{lst:decorate:overview}.

Within the bundle description document,
predicates and moves must be set off with the YAML \enquote*{tags} "!Predicate" and "!Move", respectively.
The recommended attributes\footnote{See~\autoref{task:arbitrary-attributes}.}
  for these objects \Dash \emph{required} for the graphical interface \Dash
  are as follows:
\begin{description}
\item[name] a unique name (with respect to class)\footnote{That is, I could have a predicate named \texttt{marknode} and a move named \texttt{marknode}, but I can't have two of either.} for the component
\item[description] a description of this component (\eg this predicate
  will be true if and only if the node is unmarked and isn't adjacent
  to any marked nodes).
\item[author] the author(s) of this component
\item[date] the date of the last edit (not automatically tracked; see~\autoref{tast:autodate})
\item[filename] the name of the file (without the directory) containing the definition for the component.
  That is, \directory{mark.py} rather than \directory{moves/mark.py}.%
  \footnote{Filenames \Dash as long as they are consistent \Dash are arbitrary.
    The core tool has not been tested with atypical filenames (see~\autoref{task:test}).
    It is recommended that filenames are kept to alphanumeric characters and hyphens.}
\end{description}
Per \autoref{task:texport}, there is a final \emph{recommended} attribute:
\begin{description}
\item[tex] \TeX nical documentation for the component
\end{description}


\subsection{Rules}
\label{sec:rules}

Rules are declared as items (with the "!Rule" tag)
  under the "rules" attribute of an algorithm.
They are given much the same attributes as predicates and moves,
  but with additional attributes of "predicate" and "moves"
  that map to the name of a predicate and names of moves within the bundle.

\subsection{Algorithms}
\label{sec:algorithms}

Algorithms (tagged as "!Algorithm") are simply the aggregation of their rules,
  with additional attributes for "name", "author", and "date".
%%% Local Variables: 
%%% mode: latex
%%% TeX-master: "../smp.tex"
%%% TeX-engine: xetex 
%%% TeX-PDF-mode: t 
%%% End: 
