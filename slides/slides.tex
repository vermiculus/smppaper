% arara: xelatex
% arara: biber
% arara: xelatex
% arara: xelatex
% arara: xelatex

\documentclass[
% handout,
]{beamer}

\usepackage[backend=biber, doi=false]{biblatex}
\addbibresource{refs.bib}


\usepackage{xparse}
\usepackage{mathtools}
\usepackage{amsmath}
\usepackage{smcm-math}
\usepackage{smcm-cosc-graphs}
\usepackage{csquotes}
\usepackage{listings}
\lstset{
  basicstyle=\ttfamily
}
\usepackage{fontspec}
\setmonofont{Courier}
\lstdefinelanguage{ssa}{
  mathescape=true,
  frame=none,
  morekeywords={local, if, then},
  backgroundcolor=\color{white},
  showstringspaces=false,
  showspaces=false,
  showtabs=false,
}
\def\algorithmautorefname{Algorithm}

\usepackage{tikz}
\usetikzlibrary{graphs}

\tikzset{
  marked/.style={
    color=white,
    fill=black,
  }
}


\NewDocumentCommand \BooleanTrueValue { } {
  \text{True}
}

\NewDocumentCommand \BooleanFalseValue { } {
  \text{False}
}

\NewDocumentCommand \SetBoolean { } {
  \Set{\BooleanTrueValue, \BooleanFalseValue}
}


\usetheme{Goettingen}
%\usecolortheme{spruce}

\title{Self-Stabilizing Algorithms}
\author[%
Allred \and
Bromer \and
Gladu]
{Sean Allred \and
  Shay Bromer \and
  Steven Gladu}
\date{April 29, 2014}
\institute[SMCM]{St. Mary's College}

\begin{document}
\begin{frame}
  \maketitle
\end{frame}

\section{Introduction}
\begin{frame}
  \frametitle{Introduction}
  \begin{block}{General}
    \begin{itemize}[<+->]
    \item an entire family of distributed algorithms
    \item most effectively applied to connected, \emph{sparse} networks
    \item self-stabilizing\slash fault-tolerant
    \end{itemize}
  \end{block}
  \begin{block}{Qualities}
    \begin{itemize}[<+->]
    \item each node is semi-autonomous
    \item very parallelizable
    \item execution is sometimes controlled by a central daemon
    \end{itemize}
  \end{block}
\end{frame}

\section{History}
\begin{frame}
  \frametitle{History}
  \begin{description}[<+->]
  \item[\citeyear{dew:sem}] Put forward by \citeauthor{dew:sem} in
    \citetitle{dew:sem} \autocite{dew:sem}
  \item[\citeyear{lamport}] \citeauthor{lamport} draws attention to the above,
    calling~it \enquote{Dijkstra's most brilliant work}~\autocite{lamport}
  \end{description}
\end{frame}

\section{Definition}
\begin{frame}
  \frametitle{Definition}
  \begin{quote}
    A distributed algorithm is self-stabilizing if and only if,
    starting from an arbitrary state, the algorithm will coerce the
    system into a legitimate state after a \alert<4->{finite} number of
    moves.
  \end{quote}

  \vfill

  \begin{description}
  \item[convergence]<2-> the algorithm will eventually stop
  \item[closure]<3-> the algorithm stops if and only if \\
    the system is in a legitimate state
  \end{description}
\end{frame}
\begin{frame}
  \frametitle{Definition}
  \begin{block}{Functions of a Node and its Neighbors}
    \begin{description}
    \item[predicate]<2-> $\Function[f]{v, N(\!v)}{\SetBoolean}$
    \item[move]<3-> new state for $v$
    \end{description}
  \end{block}
  \begin{block}{Collections of Rules}
    \begin{description}
    \item[algorithm]<4-> $\Set{\QuickSequence{R}}$
    \item[rule]<5-> $\Function[R_i]{P}{
        \operatorname{rand}\Set{\QuickSequence{M}}}$
    \end{description}
  \end{block}
\end{frame}

\section{Running}
\begin{frame}
  \frametitle{Running an Algorithm}
  \begin{enumerate}[<+->]
  \item System received in arbitrary state \label{step:receive}
  \item Central daemon chooses \enquote*{privileged} node $p$
  \item $p$ makes its move
  \item Process repeats; continuing to step~\autoref{step:receive}
  \end{enumerate}
\end{frame}

\section{Examples}
\begin{frame}
  \frametitle{Examples}
  \begin{description}[<+->]
  \item[Maximal Independent] A largest set of nodes $V$ that are mutually independent
  \item[Token Passing] Ensure there is only one token in a ring
    (example from \citeauthor{dew:sem}'s \citeyear{dew:sem} paper)
  \end{description}
\end{frame}
\subsection{Independent Set}
\begin{frame}[fragile]
  \frametitle{Independent Set}
  \newcommand\fOf[1]{f\,(#1\,)}
  \begin{lstlisting}[language=ssa]
local $f$

ENTER
  if $\fOf{i} = 0 \land \ForAll{j \in \OpenNeighborhood{i}}{\fOf{j} = 0}$
  then:
    $\fOf{i} = 1$

LEAVE
  if $\fOf{i} = 1 \land \Exists{j \in \OpenNeighborhood{i}}{\fOf{j} = 1}$
  then:
    $\fOf{i} = 0$
  \end{lstlisting}
\end{frame}
\begin{frame}
  \frametitle{Independent Set \qquad $t=0$}
  \centering
    \begin{tikzpicture}
      \node[graph vertex, marked] (1) at (0 , 3) {$v_1$};
      \node[graph vertex,       ] (2) at (0 , 6) {$v_2$};
      \node[graph vertex,       ] (3) at (3 , 3) {$v_3$};
      \node[graph vertex, marked] (4) at (3 , 6) {$v_4$};
      \node[graph vertex,       ] (5) at (5 , 8) {$v_5$};

      \graph {
        (1) -- (3) -- (4) -- (2) --[bend left] (5),
        (1) -- (2),
        (1) -- (4),
        (3) --[bend right] (5)
      };
  \end{tikzpicture}
\end{frame}
\begin{frame}
  \frametitle{Independent Set \qquad $t=1$}
  \centering
  \begin{tikzpicture}
      \node[graph vertex, marked] (1) at (0 , 3) {$v_1$};
      \node[graph vertex,       ] (2) at (0 , 6) {$v_2$};
      \node[graph vertex,       ] (3) at (3 , 3) {$v_3$};
      \node[graph vertex, marked] (4) at (3 , 6) {$v_4$};
      \node[graph vertex, marked] (5) at (5 , 8) {$v_5$};

      \graph {
        (1) -- (3) -- (4) -- (2) --[bend left] (5),
        (1) -- (2),
        (1) -- (4),
        (3) --[bend right] (5)
      };
  \end{tikzpicture}
\end{frame}
\begin{frame}
  \frametitle{Independent Set \qquad $t=2$}
  \centering
  \begin{tikzpicture}
      \node[graph vertex, marked] (1) at (0 , 3) {$v_1$};
      \node[graph vertex,       ] (2) at (0 , 6) {$v_2$};
      \node[graph vertex,       ] (3) at (3 , 3) {$v_3$};
      \node[graph vertex,       ] (4) at (3 , 6) {$v_4$};
      \node[graph vertex, marked] (5) at (5 , 8) {$v_5$};

      \graph {
        (1) -- (3) -- (4) -- (2) --[bend left] (5),
        (1) -- (2),
        (1) -- (4),
        (3) --[bend right] (5)
      };
  \end{tikzpicture}
\end{frame}
\subsection{Token Passing}
\begin{frame}[fragile]
\frametitle{Token Passing}
\begin{lstlisting}[language=ssa]
local $x$

$\Function[t]{\emptyset}{\exists i \in P : \forall j \neq i, x\,(j\,) = 0}$

SAURON
  if $\lnot t \land x\,(0) = x\,(n\,)$
  then:
    $x\,(0) = x\,(0) + 1 \pmod{k}$

ONERING
  if $\lnot t \land x\,(i\,) \neq x\,(i-1)$
  then:
    $x\,(i\,) = x\,(i-1)$
\end{lstlisting}
\end{frame}
\section{Time Complexity}
\begin{frame}
  \frametitle{Time Complexity}
  Complexity can be measured in \enquote*{rounds} or \enquote*{cycles}

  \vfil

  \begin{description}
  \item[round]<2-> one move is executed for one node
  \item[cycle]<3-> all privileged nodes have moved (simultaneously)
  \end{description}
\end{frame}
\section{Industry Use}
\begin{frame}
  \frametitle{Industry Use}
  \begin{description}[<+->]
  \item[BalanceMQ] \autocite{gh:balancemq}
    \enquote{continuous workload balancer} \\
    (uses algorithm from \citeauthor{lafon:balanceMQ}~\autocite{lafon:balanceMQ})
  \item[libcircle] \autocite{gh:libcircle}
    \enquote{an API for distributing [\dots] parallel workloads}
  \item[Mobile Networks] \autocite{ex:mobile3,ex:mobile2,ex:mobile1}
  \end{description}
\end{frame}
\begin{frame}
  \frametitle{References}
  \renewcommand*\bibfont{\tiny}
  \printbibliography
\end{frame}
\end{document}

%%% Local Variables:
%%% mode: latex
%%% TeX-master: t
%%% TeX-PDF-mode: t
%%% TeX-engine: xetex
%%% End:
