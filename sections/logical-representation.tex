\section{Logical Representation}
\label{sec:logic-repr}

The entirety of this tool is written in Python~3.
The following listings serve to introduce the organization of the library and
  to serve as a reference to be used when extending this library.\footnote{%
    All development is tracked as a literate program on GitHub
    at \url{http://www.github.com/vermiculus/ssa-tool}.}

\subsection{Predicate}
\label{sec:logic-repr:predicate}

According to~\eqref{eq:define-predicate},
  a predicate is defined as some function of
  a node and its neighborhood returning a Boolean value.
This is reflected in~\autoref{lst:predicate},
  where the \object{Predicate} object maintains a pure Python function
  which is stored and called with the arguments $n$ and $\OpenNeighborhood{n}$,
  respectively.
\begin{warning}
  It is assumed that the actual neighborhood is given.
  This object is intended to be used only internally,
    and all \emph{internal} functions will indeed pass it the distance-1 neighborhood.
  It is left as an unintelligent definition to enable later extension.
\end{warning}
\lstinputlisting[
float,
caption={A predicate is a function $\Function{n, \OpenNeighborhood{n}}{\SetBoolean}$},
label={lst:predicate},
linerange=predicate-endpredicate,
]{../src/README.org}

\subsection{Move}
\label{sec:logic-repr:move}

According to~\eqref{eq:define-move},
  a move is defined as a function
  $\Function{n, \OpenNeighborhood{n}}{n^\prime, \OpenNeighborhood[\prime]{n}}$.
This is directly translated into Python as a \object{Move} object in~\autoref{lst:move}.
\begin{warning}
  As with \object{Predicate}s above,
    it is assumed that the function definition matches
    the mathematical definition given in~\eqref{eq:define-move}.
  Specifically, the function \emph{must} return a tuple of
    the new node and new neighborhood as its move.
  This object is intended to be used only internally,
    and all \emph{internal} functions will indeed pass it appropriate values.
  It is left as an unintelligent definition to enable later extension.
\end{warning}
\todo[code]{Make sure this is as advertised.}%
\todo[code]{Perhaps this can be mitigated?
  Pass a simple star graph to the function to see if it returns the right thing?}%
\lstinputlisting[
float,
caption={A move is a function $\Function{n, \OpenNeighborhood{n}}
  {n^\prime, \OpenNeighborhood[\prime]{n}}$},
label={lst:move},
linerange=move-endmove,
]{../src/README.org}

\subsection{Self-Stabilizing Algorithm}
\label{sec:logic-repr:self-stab-algor}

According to~\eqref{eq:define-ssalg},
  a self-stabilizing algorithm is defined as a collection
  of these \object{Predicate} and \object{Move} objects.
These must be provided as the constructor argument "ruleset",
  as in \autoref{lst:algorithm-init}.
\lstinputlisting[
float,
caption={According to~\eqref{eq:define-ssalg},
  a self-stabilizing algorithm is exactly a
  set of rules from predicate to moves.},
label={lst:algorithm-init},
linerange=algorithm-endalgorithm,
]{../src/README.org}
In an \object{Algorithm}, the "ruleset" is assumed to have
  dictionary-like structure and behavior.
This is asserted within the constructor (as can be seen in~\autoref{lst:algorithm-type-check}),
  but perhaps the clearest definition of the structure can be given as an example.
Refer to
\lstinputlisting[
float,
caption={An example structure to use as an \object{Algorithm}'s \texttt{\small ruleset}},
label={lst:algorithm-ex},
linerange={algorithm-ruleset-ex}-{end-algorithm-ruleset-ex},
]{../src/README.org}

\subsubsection{Type Checking}
\label{sec:logic-repr:self-stab-algor:type-checking}

\todo[reword]{awkward phrasing}%
Since there is a strict definition on an algorithm's ruleset
  where neglecting to adhere to a specific format and behavior
  will cause other parts of the system to fail,
  a set of assertions is made upon "ruleset" that ensures at least non-crashing behavior.
These assertions (which can be reviewed in~\autoref{lst:algorithm-type-check}) check the following:
\begin{itemize}
\item "ruleset" is a collection that can be iterated upon directly,
\item all key-items in "ruleset" can be called directly,
\item all key-items in "ruleset" are functions of two arguments, and
\item all key-items in "ruleset" map to a collection of predicates, where
  \begin{itemize}
  \item each predicate can be called directly, and
  \item each predicate is a function of two arguments.
  \end{itemize}
\end{itemize}
It is quite possible that time could be saved by
  forgoing this initial type-check and simply assuming the behavior exists.
It is unknown to me whether dynamic error-catching (as in a "try"--"except" block)
  introduces a time constraint, where it would be far more detrimental to actual use.
\lstinputlisting[
float,
caption={Ensuring the rule-set we were given is usable as
  a definition of a self-stabilizing algorithm.},
label={lst:algorithm-type-check},
linerange={algorithm-ruleset-assertions}-{end-algorithm-ruleset-assertions},
]{../src/ssa/core/Algorithm.py}

\subsection{Central Daemon}
\label{sec:logic-repr:daemon}

This implementation's testing model is based of the concept of a \Term{central daemon}.
(This was the first model explicitly introduced
  by \citeauthor{dew:sem}~\autocite{dew:sem}.)
The \object{Algorithm} class plays double-duty to fulfill this role as a default behavior,
  but perhaps this should be altered.
\todo[code]{Should \object{Algorithm} actually know how to run itself,
  or should a separate \object{Daemon} object be created to handle this?
  Honestly, I'm leaning toward the latter.}%
The overall implementation and interface is available in~\autoref{lst:daemon-run},
  but we will look at each part individually.
\todo[code]{The algorithm should automatically stop when it has stabilized.}%
\lstinputlisting[
float=p,
caption={A generalized run of a self-stabilizing algorithm.},
label={lst:daemon-run},
linerange={daemon-run}-{end-daemon-run},
]{../src/README.org}
The daemon begins by finding assembling a list of privileged nodes.
To do this, it \emph{must} go through each node and check it for privilege.
\begin{warning}
  I consider this the greatest single bottleneck in the entire project.
  It is feasible that this can be altered into a distributed\slash threaded algorithm,
    but I haven't the knowledge to do so.
  The fact that predicates should \emph{never} write to a node or its neighbors
    allows this distribution of work among threads.
  This should be implemented in a future iteration.
\end{warning}
\lstinputlisting[
float=p,
caption={Finding privileged nodes.},
label={lst:daemon-find-privileged-nodes},
linerange={daemon-find-privileged-nodes}-{end-daemon-find-privileged-nodes},
]{../src/README.org}
Within each iteration of the loop through the nodes in the graph
  (see~\autoref{lst:daemon-find-privileged-nodes}),
  we check to see if the node is privileged or not
  (see~\autoref{lst:daemon-get-privileges}).
If it is privileged, we add the privileged node to a dictionary called "privileged_nodes".
\begin{warning}[2]
  We can also increase performance here with threading.
  Be careful of the difference, however:
    in the above we were simply reading information and
    running it through a predicate function.
  (In fact, we are doing the same here,
    but the above is a separate idea that can be sorted out.)
  The difference here is that we are altering the "privileged_nodes" data structure.
  The branch in~\autoref{lst:daemon-get-privileges} introduces a race condition:
    if two predicates $P_1, P_2$ simultaneously succeed for
    an as-yet unprivileged node $n$ at time $t$,
    time $t+1$ will result in only one unknown predicate
    $P_i$ being written to "privileged_nodes".
  Altering the data structure used for "privileged_nodes" may remove this limitation.
\end{warning}
\todo[code]{Perhaps we should just collect all predicate--move pairs in a set
  and randomly choose one?  This would be more statistically sound, but does it matter?}%

\lstinputlisting[
float=p,
caption={Getting the privileges of a single node.},
label={lst:daemon-get-privileges},
linerange={daemon-get-privileges}-{end-daemon-get-privileges},
]{../src/README.org}
\lstinputlisting[
float=p,
caption={Picking a random, satisfied predicate.},
label={lst:daemon-pick-predicate},
linerange={daemon-pick-predicate}-{end-daemon-pick-predicate},
]{../src/README.org}
\lstinputlisting[
float=p,
caption={Applying a random move enabled by the satisfied predicate.},
label={lst:daemon-apply-move},
linerange={daemon-apply-move}-{end-daemon-apply-move},
]{../src/README.org}


%%% Local Variables: 
%%% mode: latex
%%% TeX-master: "../smp.tex"
%%% TeX-PDF-mode: t 
%%% reftex-cite-format: "\\autocite{%l}" 
%%% TeX-command-default: "arara"
%%% TeX-engine: xetex
%%% End: 
