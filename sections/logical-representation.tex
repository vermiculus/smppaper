\section{Logical Representation}
\label{sec:logic-repr}

The entirety of this tool is written in Python~3.
The following listings serve to introduce the organization of the library and
  to serve as a reference to be used when extending this library.\footnote{%
    All development is tracked as a literate program on GitHub
    at \url{http://www.github.com/vermiculus/ssa-tool}.}

\subsection{Predicate}
\label{sec:logic-repr:predicate}

According to~\eqref{eq:define-predicate},
  a \gls{predicate} is defined as some function of
  a node and its neighborhood returning a Boolean value.
This is reflected in the code below,
  where the \object{Predicate} object maintains a pure Python function
  which is stored and called with the arguments $n$ and $\OpenNeighborhood{n}$,
  respectively.\footnote{%
    It is assumed that the actual neighborhood is given.
    This object is intended to be used only internally,
      and all internal functions will indeed pass it the distance-1 neighborhood.
    It is left as an unintelligent definition to enable later extension.}
\lstinputlisting[linerange=predicate-endpredicate]{../src/ssa/core/Predicate.py}

\subsection{Move}
\label{sec:logic-repr:move}

According to~\eqref{eq:define-move},
  a \gls{move} is defined as a function
  $\Function{n, \OpenNeighborhood{n}}{n^\prime, \OpenNeighborhood[\prime]{n}}$.
This is directly translated into Python as follows.
\lstinputlisting[linerange=move-endmove]{../src/ssa/core/Move.py}

\subsection{Self-Stabilizing Algorithm}
\label{sec:logic-repr:self-stab-algor}

According to~\eqref{eq:define-ssalg},
  a self-stabilizing algorithm is defined as a collection
  of these \object{Predicate} and \object{Move} objects.
\lstinputlisting[linerange=algorithm-endalgorithm]{../src/ssa/core/Algorithm.py}
These must be provided as the Python dictionary \lstinline|ruleset|,
\subsubsection{Type Checking}
\label{sec:logic-repr:self-stab-algor:type-checking}

\lstinputlisting[
float,
caption={Ensuring the rule-set we were given is usable as
  a definition of a self-stabilizing algorithm.},
label={lst:algorithm-type-check},
linerange={algorithm-ruleset-assertions}-{end-algorithm-ruleset-assertions},
]{../src/ssa/core/Algorithm.py}

\subsection{Central Daemon}
\label{sec:logic-repr:daemon}

This implementation's testing model is based of the concept of a \Term{central daemon}.
(This was the first model explicitly introduced by \citeauthor{Dijkstra:1974:SSS:361179.361202}~\autocite{Dijkstra:1974:SSS:361179.361202}.)

%%% Local Variables: 
%%% mode: latex
%%% TeX-master: "../smp.tex"
%%% TeX-PDF-mode: t 
%%% reftex-cite-format: "\\autocite{%l}" 
%%% End: 
