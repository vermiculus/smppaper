\section{Logical Representation}
\label{sec:logic-repr}

The entirety of this tool is written in and tested with Python~3.3.5.
The following listings serve to introduce the organization of the library and
  to serve as a reference to be used when extending it.\footnote{%
    All development is tracked as a literate program on GitHub
    at \url{http://www.github.com/vermiculus/ssa-tool}.}

\subsection{Predicate and Moves}
\label{sec:logic-repr:predicate-move}

According to~\autoref{eq:define-predicate} and~\autoref{eq:define-move},
  both predicates and moves are defined as simple functions of
  a node and its neighbors, where
  each predicate returns a Boolean value and
  each move returns an updated node and neighborhood.
However, in order for the tool to be useful in a collaborative setting,
  there needs to be both documentation
  and at least a few pieces of other metadata attached to these basic elements.
The population of this metadata is the responsibility of two distinct classes
  \Dash \object{Predicate} and \object{Move} \Dash
  that provide fields for the following information:
  \begin{description}
  \item[author] author(s) of the function
  \item[date] timestamp for editing
  \item[name] short summary of function\footnote{Must be unique; see~\autoref{sec:bundle-descr-docum}.}
  \item[\TeX nical documentation] mathematical definition
  \item[description] long form discussion of the function
  \end{description}

\paragraph{Definition and Loading}
As a bundle is designed to be
  a self-sufficient representation
  of one or more algorithms,
  all predicate and move definitions are stored locally
  as Python~3 function bodies within the
  \directory{predicate} and \directory{move} directories
  under root.
For example, if we wished to create a predicate
  to determine if a node could mark itself
  (considering the \Algorithm{IndSet} problem),
  we could put~\autoref{lst:logic:pred-def} in
  \directory{predicates/my-pred.py}.
\begin{lstlisting}[
float,
caption={A model definition for a predicate:
  $\ForAll{n \in \ClosedNeighborhood{v}}
          {\text{$n$ is not marked}}$.},
label={lst:logic:pred-def},
]
if v['marked']:
  return False
for neighbor in N:
  if n['marked']:
    return False
return True
\end{lstlisting} % unmarked-and-neighbors-unmarked.py
The contents of this file will be evaluated as
  a function of three arguments:
\begin{description}
\item[\texttt{self}] The current bundle.
  It is not recommended that this variable is used,
    but it exists and is a reserved word.
  Do not alter this variable unless you are confident in your actions.
\item[\texttt{v}] The current node ("v" for \enquote*{vertex}).
  Its data may be accessed dictionary-style as in~\autoref{lst:logic:pred-def}.
\item[\texttt{N}] The neighborhood of "v" as a Python~3 "set".
  Each member may be viewed as the above.
\end{description} % description of arguments self, v, and N
\begin{warning}
  Do not alter the values of "v" or "N" in a predicate function.
  In future versions these values may be copied
    before they are made available to the function body (see~\autoref{task:valid-pred}),
    but no such provisions are currently being made.
  Altering them will introduce race conditions in execution.
\end{warning} % don't alter v or N
\begin{warning}
  You may view the contents of "self" if you so desire
    (with the recognition that this potentially invalidates
    the definition of a self-stabilizing algorithm),
  but under no circumstances should you \emph{ever} alter its contents.
  Doing so will almost certainly compromise the logical consistency
    of the bundle and could irreversibly destroy all unsaved work
    in the current bundle.
\end{warning} % don't alter self. really though.
\begin{warning}
  Since Python is a whitespace-sensitive language,
    all function body definitions \emph{must} be indented
    using four spaces.
  The use of tabs (or inconsistent use of spaces)
    will cause the parser to crash at~\autoref{lst:decorate:overview}.
\end{warning} % whitespace sensitive language

\begin{lstlisting}[float, caption={Predicates and moves are defined by named files in the bundle.  See also~\autoref{task:smarter-def}.}, label={lst:decorate:overview}]
def load_definition(self, path, obj):
  if hasattr(obj, 'filename'):
    <<inherit class variables for YAML export>>

    # read definition file
    with open('/'.join([path, obj.ssa_folder, obj.filename])) as f:
      lines = f.readlines()

    # store the definition in a 'hidden' variable
    obj._definition = lines

    # prepare a string to execute as a definition
    lines = ['def temp(self, v, N):\n'] + \
            ['    ' + l for l in lines]

    # execute the string, defining the 'temp' function
    exec(''.join(lines), locals())

    # set this function as the behavior for the singleton class
    ssa_obj.__class__.__call__ = locals()['temp']
\end{lstlisting}

\subsection{Self-Stabilizing Algorithm}
\label{sec:logic-repr:self-stab-algor}

According to~\autoref{eq:define-ssalg},
  a self-stabilizing algorithm is defined as a collection
  \Dash actually a \emph{mapping} \Dash 
  of these predicates to a set of their potential moves.
These must be provided as the constructor argument "ruleset",
  as in \autoref{lst:algorithm-init}.
\begin{lstlisting}[float, caption={Each rule must be resolved \emph{after} all predicates and moves have been defined.}, label={lst:algorithm-init}]
mapping = {entity.name if hasattr(entity, 'name')
                       else repr(entity)          : entity
           for entity in entities}
for rule in self.rules:
    rule.predicate = mapping[rule.predicate]
    rule.moves = [mapping[m] for m in rule.moves]
\end{lstlisting}

In an \object{Algorithm}, the "ruleset" is assumed to have
  dictionary-like structure and behavior.
A single predicate maps to a "list" or "set" of potential moves, as in~\autoref{eq:define-ssalg}.
Refer to \autoref{lst:algorithm-ex} for an example of this data structure.
\begin{lstlisting}[float, caption={An example structure to use as an \object{Algorithm}'s \texttt{\small ruleset}}, label={lst:algorithm-ex}]
ruleset = {
    some_predicate: [some_move, some_move],
    some_predicate: [some_move, some_move, some_move],
    some_predicate: [some_move]
}
\end{lstlisting}
In addition to basic constraints on overall structure,
  every single basic component in the structure
  (each predicate and each move)
  must be directly callable with two arguments (a vertex and its neighborhood).
If both of these constraints are met, this data structure can logically be used as the algorithm's definition.

\subsection{Central Daemon}
\label{sec:logic-repr:daemon}

This implementation's testing model is based of the concept of a \Term{central daemon}.
(This was the first model explicitly introduced
  by \citeauthor{dew:sem}~\autocite{dew:sem}; see also~\autoref{sec:math-define:self-stab-algor}.)
The \object{Algorithm} class plays double-duty to fulfill this role as a default behavior,
  but perhaps this should be altered.
The overall implementation and interface is available in~\autoref{lst:daemon-run},
  but we will look at each part individually.

\begin{lstlisting}[float=p, caption={A generalized run of a self-stabilizing algorithm.}, label={lst:daemon-run}]
def run(self, graph, count=1):
    """Run the algorithm `count` times.

    <<run documentation>>
    """
    assert count >= 0

    history = list()
    while count > 0:
        <<run once>>
        count -= 1
    return history
\end{lstlisting}

The daemon begins by finding assembling a list of privileged nodes.
To do this, it \emph{must} go through each node and check it for privilege.
\begin{warning}
  I consider this the greatest single bottleneck in the entire project.
  It is feasible that this can be altered into a distributed\slash threaded algorithm,
    but I haven't the knowledge to do so.
  The fact that predicates should \emph{never} write to a node or its neighbors
    allows this distribution of work among threads.
  This should be implemented in a future iteration \Dash see~\autoref{task:stackless}.
\end{warning}

\begin{lstlisting}[float=p, caption={Finding privileged nodes.}, label={lst:daemon-find-privileged-nodes}]
for node in graph:
    neighborhood = Algorithm.neighbor_data(graph, node)
    <<determine if node is privileged>>
\end{lstlisting}

Within each iteration of the loop through the nodes in the graph
  (see~\autoref{lst:daemon-find-privileged-nodes}),
  we check to see if the node is privileged or not
  (see~\autoref{lst:daemon-get-privileges}).
If it is privileged, we add the privileged node to a dictionary called "privileged_nodes".
\begin{warning}
  We can also increase performance here with threading (see~\autoref{task:stackless}).
  Be careful of the difference, however:
    in the above we were simply reading information and
    running it through a predicate function.
  (In fact, we are doing the same here,
    but the above is a separate idea that can be sorted out.)
  The difference here is that we are altering the "privileged_nodes" data structure.
  The branch in~\autoref{lst:daemon-get-privileges} introduces a race condition:
    if two predicates $P_1, P_2$ simultaneously succeed for
    an as-yet unprivileged node $n$ at time $t$,
    time $t+1$ will result in only one unknown predicate
    $P_i$ being written to "privileged_nodes".
  Altering the data structure used for "privileged_nodes" may remove this limitation.
\end{warning}
\todo[code]{Perhaps we should just collect all predicate--move pairs in a set
  and randomly choose one?  This would be more statistically sound, but does it matter?}

\begin{lstlisting}[float=p, caption={Getting the privileges of a single node.}, label={lst:daemon-get-privileges}]
for predicate in self.ruleset:
    if predicate(graph.node[node], neighborhood.values()):
        if node in privileged_nodes:
            privileged_nodes[node] += predicate
        else:
            privileged_nodes[node] = [predicate]
\end{lstlisting}

\begin{lstlisting}[float=p, caption={Picking a random, satisfied predicate.}, label={lst:daemon-pick-predicate}]
node = random.choice(list(privileged_nodes.keys()))
neighborhood = Algorithm.neighbor_data(graph, node)
satisfied_predicate = random.choice(privileged_nodes[node])
\end{lstlisting}

\begin{lstlisting}[float=p, caption={Applying a random move enabled by the satisfied predicate.}, label={lst:daemon-apply-move}]
old_node = copy.deepcopy(node)
old_node_data = copy.deepcopy(graph.node[node])
old_neighborhood = copy.deepcopy(neighborhood)

next_move = random.choice(self.ruleset[satisfied_predicate])
next_move(graph.node[node], neighborhood)
\end{lstlisting}


%%% Local Variables: 
%%% mode: latex
%%% TeX-master: "../smp.tex"
%%% TeX-PDF-mode: t 
%%% reftex-cite-format: "\\autocite{%l}" 
%%% TeX-command-default: "arara"
%%% TeX-engine: xetex
%%% End: 
