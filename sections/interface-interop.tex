\section{Interoperability}
\label{sec:iface-interoperability}
\subsection{Native Format}
\label{sec:interop:native}
An "ssax" document is in fact a simple directory
  (see~\autoref{fig:interop:ssax-structure})
  of all files necessary to recreate and run
  every algorithm included in the bundle.
The directory is dominated by a bundle description document
  that depicts the structure of the bundle.
This description document is a YAML file
  (which \emph{must} be called "bundle.yaml")
  to allow for easy manual editing.
A full and formal description of the format is given in~\autoref{sec:bundle-descr-docum},
  but an example description document for \Algorithm{IndSet} is given in~\autoref{lst:interop:yaml-alg}.

There are several graph formats already in existence.
As this project uses NetworkX~\autocite{hagberg:networkx} for its internal representation,
  it inherits all of the abilities of this library.
Several input and output formats are available,
  detailed in~\autoref{tab:interop:import} and~\autoref{tab:interop:export} respectively.
\begin{table}
  \centering
  \providecommand\yes{\ding{51}}
  \providecommand\no{\ding{55}}
  \begin{tabular}{rcccc}
    \toprule
    Format  & Predicate & Move & Graph & Algorithm \\
    \midrule
    Text    & \yes      & \yes & \yes  & \yes      \\
    GraphML & \no       & \no  & \yes  & \no       \\
    GEXF    & \no       & \no  & \yes  & \no       \\
    PDF     & \yes      & \yes & \yes  & \yes      \\
    YAML    & \yes      & \yes & \yes  & \yes      \\
    \bottomrule
  \end{tabular}
  \caption{Import possibilities}
  \label{tab:interop:import}
\end{table}
\begin{table}
  \centering
  \providecommand\yes{\ding{51}}
  \providecommand\no{\ding{55}}
  \begin{tabular}{rcccc}
    \toprule
    Format  & Predicate & Move & Graph & Algorithm \\
    \midrule
    Text    & \yes      & \yes & \yes  & \yes      \\
    GraphML & \no       & \no  & \yes  & \no       \\
    GEXF    & \no       & \no  & \yes  & \no       \\
    PDF     & \yes      & \yes & \yes  & \yes      \\
    YAML    & \yes      & \yes & \yes  & \yes      \\
    \bottomrule
  \end{tabular}
  \caption{Export possibilities}
  \label{tab:interop:export}
\end{table}

%%% Local Variables: 
%%% mode: latex
%%% TeX-master: "../smp.tex"
%%% TeX-PDF-mode: t 
%%% reftex-cite-format: "\\autocite{%l}" 
%%% TeX-command-default: "arara"
%%% TeX-engine: xetex
%%% End: 
