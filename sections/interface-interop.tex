\section{Interoperability}
\label{sec:iface-interoperability}
Since "ssa-tool" deals with many different entities (see~\autoref{tab:format-spec}),
  there is only one formal file type that it deals with natively: XML.
This single format (extension ".ssa") branches of into several \Term{sub-formats}
  that are controlled by the format specifiers listed in~\autoref{tab:format-spec}.
Formal schemas for these are available in the sources,
  but I produce a simple example of the "predicate" format in~\autoref{lst:interop:xml-pred}.
\begin{lstlisting}[
language=xml,
caption={An example predicate in XML format},
alsodigit={-}
label={lst:interop:xml-pred},
morekeywords={ssa:save,ssa:format,ssa:predicate,ssa:function,ssa:human-documentation,ssa:texnical-documentation,ssa:function-body,name,function-name,node-arg,neighborhood-arg},
]
<ssa:save>
  <ssa:format version="1.0" entity="predicate" />
  <ssa:predicate name="All Neighbors Unmarked"
        function-name="all_neighbors_unmarked"
             node-arg="this_node"
     neighborhood-arg="neighbors">
    <ssa:texnical-documentation>
      \forall u in N(v) "marked"(u) = 0
    </ssa:texnical-documentation>
    <ssa:human-documentation>
      Each node in the neighborhood of
      the current node is unmarked.
    </ssa:human-documentation>
    <ssa:function-body>
      return all(map(lambda u: not u['marked'],
                     neighbors))
    </ssa:function-body>
  </ssa:predicate>
</ssa:save>
\end{lstlisting}
\begin{warning}[2]
  This file format may introduce bugs where the function documentation includes valid XML.
  Work \emph{must} be done to either mitigate this risk or remove it entirely.
\end{warning}

\begin{table}
  \centering
  \begin{tabular}{r>{\ttfamily\small entity=\char`"}l<{\char`"}}
    \toprule
    Entity    & format    \\
    \midrule
    Predicate & predicate \\
    Move      & move      \\
    Graph     & graph     \\
    Algorithm & algorithm \\
    Test      & algtest   \\
    \bottomrule
  \end{tabular}
  \caption{Entity specifiers of the native SSA format}
  \label{tab:format-spec}
\end{table}

There are several graph formats already in existence.
As this project uses NetworkX~\autocite{hagberg:networkx} for its internal representation,
  it inherits all of the abilities of this library.
Several input and output formats are available.
\begin{table}
  \centering
  \providecommand\yes{\ding{51}}
  \providecommand\no{\ding{55}}
  \begin{tabular}{rcccc}
    \toprule
    Format  & Predicate & Move & Graph & Algorithm \\
    \midrule
    Text    & \yes      & \yes & \yes  & \yes      \\
    GraphML & \no       & \no  & \yes  & \no       \\
    GEXF    & \no       & \no  & \yes  & \no       \\
    PDF     & \yes      & \yes & \yes  & \yes      \\
    \bottomrule
  \end{tabular}
  \caption{Import possibilities}
  \label{tab:interop:import}
\end{table}
\begin{table}
  \centering
  \providecommand\yes{\ding{51}}
  \providecommand\no{\ding{55}}
  \begin{tabular}{rcccc}
    \toprule
    Format  & Predicate & Move & Graph & Algorithm \\
    \midrule
    Text    & \yes      & \yes & \yes  & \yes      \\
    GraphML & \no       & \no  & \yes  & \no       \\
    GEXF    & \no       & \no  & \yes  & \no       \\
    PDF     & \yes      & \yes & \yes  & \yes      \\
    \bottomrule
  \end{tabular}
  \caption{Export possibilities}
  \label{tab:interop:export}
\end{table}


\subsection{Import}
\label{sec:interop:import}

\begin{figure}
  \centering
  \includegraphics{example-image-a}
  \caption{Importing an algorithm test (packaged as algorithm + graph)}
  \label{fig:iface:alg-test-import}
\end{figure}
\begin{figure}
  \centering
  \includegraphics{example-image-b}
  \caption{Exporting an algorithm test (packaged as algorithm + graph)}
  \label{fig:iface:alg-test-export}
\end{figure}

\subsection{Export}
\label{sec:interop:export}



%%% Local Variables: 
%%% mode: latex
%%% TeX-master: "../smp.tex"
%%% TeX-PDF-mode: t 
%%% reftex-cite-format: "\\autocite{%l}" 
%%% TeX-command-default: "arara"
%%% End: 
