\section{Further Work}
\label{sec:further-work}

There are many things that, due to the sheer time constraints of the project,
  have been left undone.
Some are tasks that I consider to be bugs or areas of severe lacking,

\subsection{Remaining Tasks}
\label{sec:tasks:bugs}

\begin{task}
  \label{task:test}
  Leverage the Nose system~\autocite{nose} to automate tests.
  Randomized graph construction tools have been developed in the repository
    (under \directory{simulation/generators});
  these will be helpful.
  Several tests have already been written for the generator itself \Dash
    it would be best to follow the example set forth there as tests hope to be
    efficient with the larger data structures the they would create.
\end{task}

\begin{task}
  \label{task:valid-pred}
  Prohibit the permanent modification of graph entities when applying a predicate.
  This can be effected as part of~\autoref{task:gui-definition-wrapping}.
\end{task}

\begin{task}
  \label{task:testing-interface}
  There is currently no graphical interface to the algorithm testing functionality.
  This should eventually be married with the visualizer~(\autoref{task:gui-visualizer}),
    but even a simple interface to select a graph and run the test \Dash
    piping output on-screen \Dash would suffice.
\end{task}

\begin{task}
  \label{task:interface}
  Continue work on and thoroughly test the interface.
  There are likely many issues that have not yet been discovered,
    and there are certainly best-practices in human--computer interaction
    that are not being followed.
\end{task}

\begin{task}
  \label{task:gui-syntax-highlighting}
  The predicate\slash move definition area in the graphical interface
    should have some amount of syntax highlighting.
  Code may be borrowed from the IDLE tool for Python,
    which aside from syntax highlighting, also supports interactive documentation.
  Such interactive documentation may potentially take the place of the template comments.
\end{task}

\begin{task}
  \label{task:gui-definition-wrapping}
  The predicate\slash move definition area in the graphical interface
    wraps lines in a context where line-wrapping is not desirable.
  This might be solved in either (or both) of two ways:
  \begin{itemize*}
  \item providing a second, resizable dialog that consists solely of the definition area, or
  \item providing horizontal (and vertical) scroll bars to navigate the definition.
  \end{itemize*}
  I would personally hope definitions for such basic components to remain small
    (in favor of working with~\autoref{task:smarter-def}),
    but this obviously cannot be assumed.
\end{task}

\begin{task}
  \label{task:smarter-def}
  Currently, a new class is being created for every predicate and move
  object loaded from a bundle in order to accommodate their ability to
  be called.  This can be simplified by loading the definition into
  the temporary function and then setting a \enquote*{hidden} local
  function-variable (say,~"_def") to this definition.  The
  class-level "__call__" may then simply pass the call onto this
  member variable.
\end{task}

\begin{task}
  \label{task:temp-files}
  Under the current interface, frequent saves must be made in order to
  maintain state from tab to tab.  This can by mitigated with the
  automated use of temporary files.
\end{task}


\subsection{Potential Extensions}
\label{sec:tasks:ext}

\begin{task}
  \label{tast:autodate}
  In a bundle description document (\autoref{sec:bundle-descr-docum})
    and the graphical interface (~\autoref{sec:interface-ssa}),
    a date is a timestamp of the last edit.
  This can be automated, but is not currently.
  It requires tracking changes made to the bundle during the current session,
    but shouldn't be too difficult.
\end{task}

\begin{task}
  \label{task:gephi-glue}
  Gephi~\autocite{gephi} is a very popular tool used in graph theory research.
  Equipped with an extraordinarily powerful visualizer,
    it is well-suited to run simulations using advanced graphical methods
    that may not be easily achievable within PyGame~\autocite{pygame},
    what would be the standard visualizer.
  
  Gephi is written in Java and some work will need to be done
    to glue the two together, but it is perhaps a very useful tool to have
    integrated into an already-pervasive utility.
\end{task}

\begin{task}
  \label{task:anigexf}
  The core utility keeps a history of each run of an algorithm on any graph.
  Unfortunately, there are no obvious interfaces
    (within NetworkX~\autocite{hagberg:networkx} or otherwise)
    with which to export this to a standardized format.

  One such format is \fileformat{gexf}: the Graph Exchange Format.
  Notation is available to represent change over time with respect to a node,
    but there is no facility within NetworkX's exporters to use this.

  To some extent this is to be expected; the model that NetworkX uses to represent a graph
    does not seem to allow any representation of change.
  As such, a custom exporter will need to be written (perhaps using~\autocite{pygexf}) that
    can take a graph and a list of deltas and can output the correct notation.
  This notation is detailed in the official documentation~\autocite{gexf}.
\end{task}

\begin{task}
  \label{task:gui-visualizer}
  A prototype visualizer has been written for use with the tool,
    but is not yet ready for release.
  It is still available within the GitHub repository at "vermiculus/ssa-tool".
  It leverages the graph representation and layout facilities of NetworkX~\autocite{hagberg:networkx}
    and uses PyGame~\autocite{pygame} to draw the graph on-screen.
  Given the nature of Python reference types, any and all changes to the graph
    will be immediately reflected on-screen.
  This task is merely concerned with the integration of these two tools.
\end{task}

\begin{task}
  \label{task:texport}
  It would be exceptionally useful to have printed output from the
  tool that details the definition of an algorithm or set of
  algorithms within a bundle, perhaps even typeset histories (using
  \TikZ) of a run.  \TeX nical documentation is being kept to allow
  this, but serious work in text processing will need to be done in
  order to do this cleanly.
\end{task}

\begin{task}
  \label{task:bundle-graphs}
  In order to fully facilitate reproducible research
    (as reproducible as it could be,
      considering the arbitrary nature of selection),
    graphs and corresponding tests should be included as part of bundles.
\end{task}

\begin{task}
  \label{task:save-history}
  Implement a history browser within the graphical interface.
  This is a simple parsing of the data-structure described in~\autoref{sec:logic-repr},
    and may also be integrated into the visualizer from~\autoref{task:gui-visualizer}.
\end{task}

\begin{task}
  \label{task:arbitrary-attributes}
  PyYAML~\autocite{pyyaml} is able to read and write arbitrary YAML
  attributes to and from Python objects, but there is no way to explore
  any additional attributes that might exist within the file.
  It may be useful to have a sort of \enquote*{data explorer} that can
    browse and edit these attributes.
\end{task}

\begin{task}
  \label{task:magical-references}
  Some predicates and moves may eventually be defined as the aggregation of others.
  Design and implement a syntax to refer to predicates and moves defined within the bundle
    within the definition of the predicate or move itself.
  I would suggest Noweb syntax~\autocite{noweb}, but I am far from attached to it
    (\emph{especially} if the syntax is used elsewhere in Python~3).
\end{task}

\begin{task}
  \label{task:stackless}
  The tool currently implements only the centralized daemon model of execution,
    but other models exist that are far more distributed.
  (See for example~\autocite{ssa:dist}.)
  There are many real-world simulation possibilities when using a truly distributed system,
    but the current implementation is single-threaded.

  The project may be tweaked and re-targeted toward the Stackless variant of Python~\autocite{stackless}.
  Stackless supports multi-threading as a core goal, and this task would likely be easiest to complete with it.

  This runs the risk, however, of breaking NetworkX~\autocite{hagberg:networkx}, PyYAML~\autocite{pyyaml}, or PyGame~\autocite{pygame}.
\end{task}

\subsection{Further Research Now Possible}

\begin{task}
  \label{task:genetic}
  Given how it represents algorithms as a population of predicates, moves, and rules,
    "ssa-tool" could feasibly be used to genetic algorithms that generate
    self-stabilizing algorithms.
  To do so effectively would require a quick and completely autonomous
    evaluation of effectiveness as a fitness function \Dash
    perhaps how quickly the algorithm converges or if it converges at all.
\end{task}

%%% Local Variables:
%%% mode: latex
%%% TeX-master: "../smp.tex"
%%% TeX-PDF-mode: t
%%% reftex-cite-format: "\\autocite{%l}"
%%% TeX-command-default: "arara"
%%% TeX-engine: xetex
%%% End:
