\section{Reflection}
\label{sec:reflection}

This project has been long and difficult, \dots

Like several of my peers during this project,
  the first great challenge was perhaps initial restraint and discipline.
\citeauthor{brooks:mythical-man-month} described in his book \citetitle{brooks:mythical-man-month}
  a phenomenon known as the \enquote*{second-system effect}.
The phenomenon describes the woes of even the careful the software architect,
\begin{displayquote}[\cite{brooks:mythical-man-month}]
  This second is the most dangerous system a man ever designs.
  \Elide
  The general tendency is to over-design the second system,
    using all the ideas and frills that were cautiously sidetracked on the first one.
  The result, as Ovid says, is a \enquote{big pile}.
\end{displayquote}
My reflection on my general project-management decisions bring this passage to mind
  not because I added such frills, but because I fell victim to my own reckless hubris
  in believing I knew enough about the domain to begin modeling it in a program.
\todo[ask]{Find the quote that falls along the lines of \enquote*{to program is to be absolutely certain}.
  I can't really remember how it goes
  \Dash I \emph{think} it was in the Mythical Man-Month \Dash
  but it went student--teacher--?--programmer or something of the like.}
Logically representing the system without a full \Dash or even working \Dash
  knowledge of the current research and definitions set the project back at least a month, possibly longer.
Once I relented to \emph{reading} before I \emph{wrote},
  progress began to accelerate to the point of noticeable and measurable progress.

This naturally led to the second great challenge: self-motivation and discipline (again).
As I began to research the topic in earnest,
  my time was consumed for a short time with reading and absorbing information.
This was a short-lived period of near-frenzied research
  \Dash in print, online, and in-person \Dash
  where I read a wide variety of different articles that utilized or talked about fault-tolerant systems.
(This frenzy was started by reading~\citetitle{goddard:ssa--k-distance}
  by~\citeauthor{goddard:ssa--k-distance} for a separate assignment.)
Before separate time constraints won over with sheer pressure,
  I was able to more clearly see the project's place in the overall arena of research.
This clarity of purpose \Dash
  coupled with the understanding offered by the research itself \Dash
  unsurprisingly gave me the necessary tools and motivation to fully design
  the concepts and relationships of the underlying system.

\subsection{A Brief History of Time (Since September~2013)}
Everyone loves well-documented software.
Documentation is essential \Dash without it, a project has virtually no hope of getting off the ground.
As a matter of fact, the lack of documentation in another project \Dash
  Graph\# for WPF/C\#~\autocite{palesz:graphsharp} \Dash
  drove me to choose to implement the entire project in Python over the original plan of C\#.
By all accounts, C\# seemed to be a better fit for the task at hand \Dash
  create a graphical user interface for the creation and maintainence of self-stabilizing algorithms \Dash
  but the basic tools available were not documented in the slightest.
I became frustrated enough with the experience to abandon it entirely.

Taking my advisor's\dots advice\dots, I re-implemented what I had in Python.
\todo{expand, perhaps?}

\bigskip

With the logical definitions behind me and the graphical interface ahead of me,
  I have very little crucial \emph{thinking} left to do, it would seem.
I'm not saying the creation of an interface is trivial \Dash quite the contrary \Dash
  but I am saying it will require a very different throughout process to do well.
I haven't formally decided on the graphical interface I want to use yet, but Tkinter seems like the natural choice.
\todo[cite]{tkinter}
When I discovered PyGame \Dash realizing that what I needed was
  a bare-bones graphics library for the graph-drawing portion \Dash
  it was a frustrating install to say the least on my late 2009 MacBook.
\todo[cite]{pygame}
When I finally upgraded to the current model MacBook Pro,
  the install process went smoothly and quickly.

\subsection{The Grand Distraction}
Everyone loves well-documented software.
My experience with Graph\# \Dash and the eventual migration to a more convenient platform \Dash
  encouraged me to take on the largest project of my academic career in
  the most foreign \Dash but \emph{useful} \Dash actual software devlopment paradigm
  that I had encountered to date.
Writing the document as a literate program~\autocite{knuth:lit-prog}
  forced me to write good documentation for the project,
  but perhaps this level of documentation was not the best use of my time.
The first commit activity in the source code dates to 10:30pm Christmas~Eve,
  more than halfway through the year-long project.
While I don't regret the decision to switch to Python \Dash
  and I will not say I regret the use of literate programming in this project \Dash
  I should have focused far more on the actual project
  rather than the format in which it would be written.

etc\dots

%It is painfully \Dash sometimes annoyingly \Dash obvious to those who know me that \LaTeX\ is a true passion of mine.


\begin{draftvspace}{1in}
  I haven't quite completed this section \emph{yet}, \\
    but I provide a small quip on why \\
    in the poetic words of 2am in the morning.
  \begin{verse}
    Need more reflection, but I cannot reflect \\
    on a process that isn't complete quite yet. \\
    Pages to go before I sleep. \\
    Pages to go before I sleep. \\
  \end{verse}

  \begin{displayquote}[Plato]
    At the moondial's touch of 2am, everyone becomes a poet.
  \end{displayquote}
\end{draftvspace}

%%% Local Variables: 
%%% mode: latex
%%% TeX-master: "../smp.tex"
%%% TeX-PDF-mode: t 
%%% reftex-cite-format: "\\autocite{%l}" 
%%% TeX-command-default: "arara"
%%% TeX-engine: xetex
%%% End: 
