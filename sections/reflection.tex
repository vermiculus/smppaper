\section{Reflection}
\label{sec:reflection}

This project has been long and difficult.
After starting soon after my return to campus in the fall of~2013,
  the project is finally coming to a close at the time of writing.
There are many things left to be done (see~\autoref{sec:further-work}),
  but there are also many things that have been accomplished.
Some of these accomplishments are in and of the project itself,
  but the greater accomplishments \Dash
  those that will remain with and influence me \Dash
  are those practices and perspectives that I've acquired
  through the course of this capstone experience.

\subsection{Restraints per Constraints}

Like several of my peers during this project,
  the first great challenge was perhaps initial restraint and discipline.
\citeauthor{brooks:mythical-man-month} described in his book \citetitle{brooks:mythical-man-month}
  a phenomenon known as the \enquote*{second-system effect}.
The phenomenon describes the woes of even the careful the software architect,
\begin{displayquote}[\cite{brooks:mythical-man-month}]
  This second is the most dangerous system a man ever designs.
  [\dots]
  The general tendency is to over-design the second system,
    using all the ideas and frills that were cautiously sidetracked on the first one.
  The result, as Ovid says, is a \enquote{big pile}.
\end{displayquote}
My reflection on my general project-management decisions bring this passage to mind
  not because I added such frills, but because I fell victim to my own reckless hubris
  in believing I knew enough about the domain to begin modeling it in a program.
Logically representing the system without a full \Dash or even working \Dash
  knowledge of the current research and definitions set the project back at least a month, possibly longer.
Once I relented to \emph{reading} before I \emph{wrote},
  the project accelerated to the point of noticeable and measurable progress.

This naturally led to the second great challenge: self-motivation and discipline (again).
As I began to research the topic in earnest,
  my time was consumed for a short time with reading and absorbing information.
This was a short-lived period of near-frenzied research
  \Dash in print, online, and in-person \Dash
  where I read a wide variety of different articles that utilized or talked about fault-tolerant systems.
(This frenzy was started by reading~\citetitle{goddard:ssa--k-distance}
  by~\citeauthor{goddard:ssa--k-distance} for an assignment in a separate class.)
Before other time constraints won over with the sheer pressure of impending due dates,
  I was able to more clearly see the project's place in the overall arena of research.
This clarity of purpose \Dash
  coupled with the understanding offered by the research itself \Dash
  unsurprisingly gave me the necessary tools and motivation to fully design
  the concepts and relationships of the underlying system.

Thus, the core engine \Dash "ssa-tool" itself \Dash was written and re-written time and time again.
Each successive iteration took very little time when considered on its own,
  but each was significant in marking an increase in understanding.
The final engine \Dash
  including the time for all previous iterations and less the persistent aspects \Dash
  was working within a month.
As happy as I was to complete this milestone,
  what the coming months had in store would
  overturn the perspectives I had assumed
  in creating this pre-release.

\subsection{Thinking Ahead}

The project's initial goal was \enquote*{simple}:
\begin{quote}
  Create a graphical tool for the simple creation, manipulation,
    and evaluation of self-stabilizing algorithms
    while minimizing the purely technical requirement of the user.
\end{quote}
The project was to manifest itself in a fully-featured graphical user interface.

I had created a \emph{very} robust system that worked without a graphical interface.
It was (and still is) my assumption that a good system relies on an even better engine,
  and the command-line interface was that engine.
Unfortunately, it turned out that this system was not very conducive
  to interfacing with a real-time window system.
A few minor redesigns needed to be made to glue the two together,
  and the interaction is still largely imperfect
  (as the task list in~\autoref{sec:further-work} is painfully indicative of).

The more fundamental problem was the very choice of language.
For a command-line interface, Python was an excellent choice.
It was expressive and concise, file operations were simple and
  plain-text serialization was well-supported with PyYAML~\autocite{pyyaml}.
Being comfortable on the command-line and a big fan of plain-text data,
  this stage of development was \Dash for lack of a more perfect word \Dash wonderful.

Graphical interfaces in Python, however, are \emph{not} so wonderful.

If I had considered the goals of the project more seriously
  (and perhaps recognized that I \Dash with my own preferred uses \Dash was not the primary user),
  I would have begun the implementation in~C\#
  (given the wonderful interface design tools available with Visual~Studio).
This would have absolutely \emph{thrashed} portability,
  but would have made the tool far more stable in its prescribed use case.
There are several tools that would require re-implementation
  (such as the graph drawing capabilities I had written
  for PyGame~\autocite{pygame}; see~\autoref{task:gui-visualizer}),
  but more research and simple resolve would have solved this issue
  far quicker and more effective than what has come to pass.

\subsection{The Grand Distraction}
Everyone loves well-documented software.
My experience with Graph\# \Dash and the eventual migration to a more convenient platform \Dash
  encouraged me to take on the largest project of my academic career in
  the most foreign \Dash but \emph{useful} \Dash actual software development paradigm
  that I had encountered to date.
Writing the document as a literate program~\autocite{knuth:lit-prog}
  forced me to write good documentation for the project,
  but perhaps this level of documentation was not the best use of my time
  in the beginning stages of the project.
The first commit activity in the source code dates to 10:30pm Christmas~Eve,
  more than halfway through the year-long project.
While I will never regret the use of literate programming in this project \Dash
  it is this use that largely taught me what the project \emph{was} \Dash
  I should have focused far more on the actual project
  rather than the format in which it would be written and maintained.

Additionally, the desire to produce write a \enquote*{perfect} project \Dash
  following all Python conventions \Dash
  took a large portion of time.
Working in a literate document certainly did not require this.
If I so chose, it would have been inconsequential to tangle to
  one file and one file only.
The desire to make the tool easily navigable outside of this format \Dash
  while a fine desire for a second iteration of this prototype \Dash
  was a significant time-sink.

It so happened that \Dash in the production-frenzy of the final weeks of term \Dash
  I began to work solely from what began as a \enquote*{test} file.
This was not a literate environment.
While it allowed me to work far quicker in some cases
  \Dash there was no intermediate tangling step \Dash
  it was a \emph{nightmare} to keep everything organized.
A better approach would have been to work initially from these \enquote*{test} files \Dash
  no literate programming or even significant documentation \Dash
  and then transfer and \enquote*{untangle} them into the literate document,
  writing the essay-documentation for the current component while it was still familiar to me.
Without a literate backbone to encourage it, the project's source lacks the high degree of
  documentation and structure that I pride myself in producing for my other projects.
Before the project enters my portfolio of work that I am \emph{proud} of,
  this is first and foremost.\footnote{As a sidenote,
    another hindrance \Dash completely unnecessarily \Dash
    was the lack of GitHub highlighting for the literate document.
    This has thankfully been resolved as \href{https://github.com/github/markup/issues/253}{\texttt{github/markup\#253}}.}

\subsection{Special Thanks}
There are many resources, both online and personal, that have made this research project possible.
First and foremost is my advisor, Dr.\,Alan~Jamieson, who suggested this project
  and has continually guided its development through clear explanations of the underlying concepts.
Second is the research community surrounding Python, producing such useful tools
  as NetworkX~\autocite{hagberg:networkx} (logical representation),
  PyYAML~\autocite{pyyaml} (persistent representation), and
  PyGame~\autocite{pygame} (visual representation).
Finally are the many friends who graciously listened to my explanations of this research
  and often asking questions of me that I would not have asked myself,
  always enhancing my understanding of the research domain and its audience.

%%% Local Variables: 
%%% mode: latex
%%% TeX-master: "../smp.tex"
%%% TeX-PDF-mode: t 
%%% reftex-cite-format: "\\autocite{%l}" 
%%% TeX-command-default: "arara"
%%% TeX-engine: xetex
%%% End: 
