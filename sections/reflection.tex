\section{Reflection}
\label{sec:reflection}

This project has been long and difficult, \dots

Like several of my peers during this project,
  the greatest challenge was perhaps the sheer act of restraint.
\citeauthor{brooks:mythical-man-month} described in his book \citetitle{brooks:mythical-man-month}
  a phenomenon known as the \enquote*{second-system effect}.
The phenomenon describes the woes of even the careful the software architect,
\begin{displayquote}[\citeay{brooks:mythical-man-month}, p.\,55]
  This second is the most dangerous system a man ever designs.
  \Elide
  The general tendency is to over-design the second system,
    using all the ideas and frills that were cautiously sidetracked on the first one.
  The result, as Ovid says, is a \enquote{big pile}.
\end{displayquote}
My reflection on my general project-management decisions bring this passage to mind
  not because I added such frills, but because I fell victim to my own reckless hubris
  in believing I knew enough about the domain to begin modeling it in a program.
\todo{Find quote that falls along the lines of \enquote*{to program is to be absolutely certain}.
  I can't really remember how it goes
  \Dash I \emph{think} it was in the Mythical Man-Month \Dash
  but it went student--teacher--?--programmer or something of the like.}
Logically representing the system without a full \Dash or even working \Dash
  knowledge of the current research and definitions set the project back at least a month, possibly longer.
Once I relented to \emph{reading} before I \emph{wrote},
  progress began to accelerate to the point of noticeable and measurable progress.

%%% Local Variables: 
%%% mode: latex
%%% TeX-master: "../smp.tex"
%%% TeX-PDF-mode: t 
%%% reftex-cite-format: "\\autocite{%l}" 
%%% End: 
