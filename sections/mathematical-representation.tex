\section{Mathematical Definitions}
\label{sec:math-defin}

The core engine must be correct above all.
To prove its correctness, all necessary definitions will be
  presented, cited as necessary, and referred to throughout the remainder of the paper.

\subsection{Graph}
\label{sec:math-defin:graphs}

A graph is a mathematical construct
  consisting of vertices $V$ and edges $E$ between them: $G = (V, E)$.
\done[don't need to worry about it]
\todo[ask]{graphs are also rep'd as sets of edges $v_1v_2$, but how are orphans notated? do I even need to worry about this?}
We define its open neighborhood of $v \in G$ to be
  the set of all nodes $u$ such that $v$ is adjacent to $u$:
  $\OpenNeighborhood{n} = \Set{v}[(n, v) \in E]$
  and the closed neighborhood is defined as $v$ together with its neighbors:
  $\ClosedNeighborhood{n} = \Union{\OpenNeighborhood{n}, \Set{n}}$.

\subsection{Self-Stabilizing Algorithm}
\label{sec:math-define:self-stab-algor}

As defined by \citeauthor{dew:sem}~\autocite{dew:sem},
  \done[Looks so much better this way]
  \todo{\texttt{citefield} the author}
  a self-stabilizing algorithm is a contruct composed of
  predicates and corresponding moves.
Both are functions of a single node and its neighbors.\footnote{%
  Note that this is not the same as the closed neighborhood of a node.
  Consider these as functions $\Function{n, \OpenNeighborhood{n}}{\Set ?}$.}
\Term{Predicates} are functions to the Boolean values~\eqref{eq:define-predicate},
  where each \Term{moves} updates the state of the system~\eqref{eq:define-move}.
When a predicate evaluates to $\BooleanTrueValue$ for any node $n$,
  it is said that $n$'s privilege is present.
\begin{align}
  \label{eq:define-predicate}
  \Function*[P]{n, \OpenNeighborhood{n}}{\SetBoolean},
  \\
  \label{eq:define-move}
  \Function*[M]{n, \OpenNeighborhood{n}}{n^\prime, \OpenNeighborhood[\prime]{n}}.
\end{align}

A self-stabilizing algorithm can thus be seen as
  a collection of these predicate--move pairs.
Formally, we may define such an algorithm $S$ to be a discrete function
  from predicates to a random move functions:
  \begin{equation}
    \label{eq:define-ssalg}
    \Function[S]{P}{\operatorname{RandomChoice}(\Set{M_1, M_2, \dots, M_N})}
  \end{equation}
\done[Nope! Not really]
\todo[ask]{Is there a better way of describing random choice?
  Any conventions that are set, notationally speaking?}
It was defined in this way to clarify the logical representation of the algorithm;
  see Section~\ref{sec:logic-repr:self-stab-algor}.
To distinguish from other uses of \Term{algorithm},
  this particular data structure will be called the \Term{rule set}.

Two properties of the algorithm must be shown
  for a self-stabilizing algorithm to be proven correct:~\autocite{arora:closure-and-convergence}
\begin{description}
\item[convergence] The algorithm must complete using a finite number of moves.
\item[closure] If the algorithm completes, the system is in a correct state.
  \done[closure and convergence are \emph{it}]
  \todo[ask]{Determine the qualities of the algorithm that must be shown.}
\end{description}

A self-stabilizing algorithm converges if and only if $\Exists{N \in \Naturals}{\ForAll{n > N, p \in P^n, v \in V}{p(v) = 0}}.$
\todo[ask]{should this be a citation or a lemma?}
\begin{comment}
  \label{eq:define-ssalg:converge}
  \Exists{N \in \Naturals}{\ForAll{n > N, p \in P^n, v \in V}{p(v) = 0}}.
\end{comment}
That is, after some finite number of moves $N \in \Naturals$,
  no predicate $p \in P$ privileges any vertex $v \in G$.
\todo[ask]{Is there a mathematical way of expressing repeated application like this?
  Ask Susan or Sandy or one of the math profs if Alan doesn't know.}
If there are no privileged nodes, no node can make a move and
  the system is considered \Term{stable}~\autocite{dew:sem}.

A self-stabilizing algorithm satisfies closure if and only
  the absence of a privileged node necessarily implies a correct overall state:
\begin{equation}
  \label{eq:define-ssalg:closure}
  \Exists{N \in \Naturals}{\ForAll{n > N, p \in P^n, v \in V}{p(v) = 0}}
  \implies \text{$G$ is in a correct state}.
\end{equation}

\paragraph{Example Algorithm and Run}
A popular problem in graph theory is \Algorithm{IndSet} \Dash
  the maximal independent set \Dash
  and it has a particularly simple self-stabilizing algorithm (see~\autoref{alg:ind-set}).
For clarity, the maximal independent set is defined as the set with the greatest cardinality
  within the power~set of the vertex~subsets $H \subseteq G$ such that
  no two nodes in $H$ are adjacent in $G$.

\begin{algorithm}[float]
  \caption{Maximal Independent Set, definition from \autocite{goddard:ssa--k-distance}}
  \label{alg:ind-set}
\begin{lstlisting}[language=ssa]
local $f$

ENTER
  if $f(i) = 0 \land \ForAll{j \in \OpenNeighborhood{i}}{f(j) = 0}$
  then:
    $f(i) = 1$

LEAVE
  if $f(i) = 1 \land \Exists{j \in \OpenNeighborhood{i}}{f(j) = 1}$
  then:
    $f(i) = 0$
\end{lstlisting}
\end{algorithm}

Considering~\autoref{alg:ind-set},
  the first rule simply says if some node $i$ is marked
  (where $f$ is a node-indexed Boolean array)
  \emph{and} no neighboring nodes $j$ are marked,
  then we \enquote*{enter} the set by setting our value in $f$ to 1.
Similarly, the second rule (\enquote*{leave})
  applys when, for some node $i$, the node is marked and a neighbor is as well.

The algorithm is thereby defined and is ready to be applied to
  some graph $G$ and nodes $i$ and neighbors $j$.
Consider the graph in~\autoref{fig:math-run:0}.
\begin{figure}
  \centering
  \includegraphics{example-image-a}
  \caption{An arbitrarily marked graph $G$ at time $t = 0$.}
  \label{fig:math-run:0}
\end{figure}
Nodes have been marked arbitrarily,
  representative of a system that is in a possibly incorrect state.
We pick a node \emph{arbitrarily}, in this case $v_0$,
  and apply the predicates from~\autoref{alg:ind-set} to the node
  to see which privileges are present.
The only present privilege is \enquote*{enter},
  so we execute the move prescribed by the rule and mark $v_0$: $f(v_0)=1$.
The graph from~\autoref{fig:math-run:0} is now~\autoref{fig:math-run:1},
  and the process begins again, perhaps now $v_4$ is picked and
  satisfies the \enquote*{leave} predicate and, with the privilege present,
  executes its moves.
\begin{figure}
  \centering
  \includegraphics{example-image-b}
  \caption{The graph from \autoref{fig:math-run:0} at time $t = 1$.}
  \label{fig:math-run:1}
\end{figure}

\paragraph{Different Models}
The paradigm given above is known as the central-daemon model~\autocite{dew:sem}.
At least one other model exists where each node acts
  of its own accord distinct from any central daemon.
\todo[cite]{Find a resource that talks about non-daemonized algorithm executions in-depth}
To simplify initial development, the daemon model was implemented
  as opposed to any truly distributed model.
An extension may later be added, perhaps using the Python~variant Stackless~\autocite{stackless},
  that implements an alternative run-mode that uses distributed techniques.

%%% Local Variables: 
%%% mode: latex
%%% TeX-master: "../smp.tex"
%%% TeX-PDF-mode: t 
%%% reftex-cite-format: "\\autocite{%l}" 
%%% TeX-engine: xetex 
%%% TeX-command-default: "arara"
%%% End: 
