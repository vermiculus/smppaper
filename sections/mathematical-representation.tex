\section{Mathematical Definitions}
\label{sec:math-defin}

\subsection{Graphs}
\label{sec:math-defin:graphs}

A \gls{graph} is a mathematical construct
  consisting of \glspl{node} (or \Term{vertices}) and \glspl{edge} between them:
  \[ G = (V, E). \]
\todo{graphs are also rep'd as sets of edges $v_1v_2$, but how are orphans notated?}
Graphs are used to represent networks:
\todo{draw a network of hospitals and data centers with tikz}

When any node $n$ is considered, we define its \gls{open neighborhood} to be
  the set of all nodes $n^\prime$ such that $n$ is connected (or \Term{adjacent} to) $n^\prime$:
  \[ \OpenNeighborhood{n}[G] = \Set{n^\prime}[nn^\prime \in G] \]
  and the \gls{closed neighborhood} as $n$ with its neighbors:
  \[ \ClosedNeighborhood{n}[G] = \OpenNeighborhood{n}[G] \cup \Set{n}. \]

\subsection{Finite State Machines}
\label{sec:math-defin:fsm}

A \gls{finite state machine} is a mathematical model of computation
  described by a finite number of states and transitions between those states:
  \[ M = (Q, \delta). \]
These can be represented as graphs where nodes are states in $Q$ and
  transitions are directed edges between them, as in \autoref{fig:math-defin:fsm:ex}.
  \begin{figure}
    \centering
    \begin{tikzpicture}
      [->,shorten >=1pt,node distance=3cm,auto,inner sep=2pt,>=stealth,thick]
      \node[state,initial]   (start)                         {$q_i$};
      \node[state]           (q_0)    [above right of=start] {$q_0$};
      \node[state]           (q_1)    [      right of=q_0]   {$q_1$};
      \node[state]           (q_2)    [      right of=q_1]   {$q_2$};
      \node[state,accepting] (q_3)    [      right of=q_2]   {$q_3$};
      \node[state]           (q_0p)   [below right of=start] {$q_0'$};
      \node[state]           (q_1p)   [      right of=q_0p]  {$q_1'$};
      \node[state,accepting] (q_2p)   [      right of=q_1p]  {$q_2'$};
      \node[state]           (q_fail) [above right of=q_2p]  {$q_\text{fail}'$};

      \path[->]
      (start)
      edge              node {$\varepsilon$} (q_0)
      edge              node {$\varepsilon$} (q_0p)
      (q_0)
      edge [loop above] node {0}             ()
      edge              node {1}             (q_1)
      (q_1)
      edge [loop above] node {0}             ()
      edge              node {1}             (q_2)
      (q_2)
      edge [loop above] node {0}             ()
      edge              node {1}             (q_3)
      (q_3)
      edge [loop above] node {0,1}           ()
      (q_0p)
      edge [bend left ] node {0}             (q_fail)
      edge              node {1}             (q_1p)
      (q_1p)
      edge [bend left ] node {0}             (q_2p)
      edge [loop above] node {1}             ()
      (q_2p)
      edge [loop above] node {0}             ()
      edge [bend left ] node {1}             (q_1p)
      (q_fail)
      edge [loop below] node {0,1}           ();
    \end{tikzpicture}
    \caption{An example of a finite-state machine.}
    \label{fig:math-defin:fsm:ex}
  \end{figure}

\subsection{Self-Stabilizing Algorithm}
\label{sec:math-defin:self-stab-algor}

A self-stabilizing algorithm is, as defined by
  Dijsktra \autocite{Dijkstra:1974:SSS:361179.361202},
  \todo{\texttt{citefield} the author}
  a contruct composed of \glspl{privilege} and corresponding \glspl{action}.
Both are functions of a single node and its neighbors.\footnote{%
  Note that this is not the same as the closed neighborhood of a node.}
Priveleges are functions to the Boolean values,
  where each action updates the state of the system.
(\Term{updating the system} may be realized as the function returning a new graph.)
\[
\Function[P]{G, n}{\SetBoolean}
\qquad
\Function[A]{G, n}{G^\prime}
\]

%%% Local Variables: 
%%% mode: latex
%%% TeX-master: "../smp.tex"
%%% TeX-PDF-mode: t 
%%% reftex-cite-format: "\\autocite{%l}" 
%%% End: 
