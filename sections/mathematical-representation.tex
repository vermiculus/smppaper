
\section{Mathematical Definitions}
\label{sec:math-defin}

\subsection{Graphs}
\label{sec:graphs}

A \gls{graph} is a mathematical construct
  consisting of \glspl{node} (or `vertices') and \glspl{edge} between them:
  \[ G = (V, E). \]
\todo{graphs are also rep'd as sets of edges $v_1v_2$, but how are orphans notated?}
Graphs are used to represent networks:
\todo{draw a network of hospitals and data centers with tikz}

When any node $n$ is considered, we define its \gls{open neighborhood} to be
  the set of all nodes $n^\prime$ such that $n$ is connected (or `adjacent' to) $n^\prime$:
  \[ \OpenNeighborhood{n}[G] = \Set{n^\prime}[nn^\prime \in G] \]
  and the \gls{closed neighborhood} as $n$ with its neighbors:
  \[ \ClosedNeighborhood{n}[G] = \OpenNeighborhood{n}[G] \cup \Set{n}. \]

\subsection{Finite State Machines}
\label{sec:fsm}

A \gls{finite state machine} is a mathematical model of computation
  described by a finite number of states and transitions between those states:
  \[ M = (Q, \delta). \]
These can be represented as graphs where nodes are states in $Q$ and
  transitions are directed edges between them:
  \todo{use tikz automata to draw a graph here;
    plenty of examples in old theory homework}

\subsection{Self-Stabilizing Algorithm}
\label{sec:self-stab-algor}

A self-stabilizing algorithm is, as defined by
  Dijsktra \autocite{Dijkstra:1974:SSS:361179.361202},
  \todo{\texttt{citefield} the author}
  a contruct composed of \glspl{privilege} and corresponding \glspl{action}.
Both are functions of a single node and its neighbors.
Predicates are functions to the Boolean values,
  where each action returns a new graph.

%%% Local Variables: 
%%% mode: latex
%%% TeX-master: "../smp.tex"
%%% TeX-PDF-mode: t 
%%% End: 
