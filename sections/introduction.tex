\section{Introduction}
\label{sec:introduction}
\subsection{Domain Introduction}
In his seminal paper on the topic~\autocite{dew:sem},
  \citeauthor{dew:sem} introduced the concept of
  self-stabilizing algorithms.
He visualized the realization of these algorithms as taking advantage of
  a very specific scenario:
\begin{displayquote}[\citeay{dew:sem}]
  We consider a connected graph in which
    the majority of the possible edges are missing and
    a finite state machine is placed at each node;
    machines placed in directly connected nodes are called each other's neighbors.
  For each machine one or more so-called \enquote{privileges} are defined,
    \ie Boolean functions of its own state and the states of its neighbors;
    when such a Boolean function is true,
    we say that the privilege is \enquote{present}.
  In order to model the undefined speed ratios of the various machines,
    we introduce a central daemon \Dash
    its replacement by a distributed daemon falls outside the scope of this article \Dash
    that can \enquote{select} one of the privileges present.
  The machine enjoying the selected privilege will then make its \enquote{move};
    \ie it is brought into a new state that is a function of
    its old state and the states of its neighbors.
  If for such a machine more than one privilege is present,
    the new state may also depend on the privilege selected.
  After completion of the move, the daemon will select a new privilege.
\end{displayquote}

\subsection{Consequences and Further Research}
This canonical definition is inherently limited, however;
  each node $n$ only knows the state of its distance-1 neighbors $v \in \ClosedNeighborhood{n}$.
\citeauthor{gairing:distance-2} introduced an approach for
  simulating distance-2 information~\autocite{gairing:distance-2},
  which was later generalized to simulating distance-$k$
  information~\autocite{goddard:ssa--k-distance} by~\citeauthor{goddard:ssa--k-distance},
  necessarily introducing appropriate polynomial time and space complexities.
\citeauthor{dew:sem}'s restrictive-yet-practical definition
  of a self-stabilizing algorithm
  was now \emph{far} less restrictive,
  since \citeauthor{goddard:ssa--k-distance} allowed \emph{any} distance-$k$ algorithm
  to be simulated in the original definition within a reasonable complexity increase.

Since then, many such algorithms have been put forward, unsurprisingly.
Talks such as~\autocite{dolev:talk} present both
  the many applications of self-stabilization algorithms and
  the research done on the algorithms in general.
Given their innately parallel nature,
  self-stabilizing algorithms are \emph{incredibly} useful
  for accurately modeling and solving real-world problems.

\subsection{Related Work}
\label{sec:introduction:related-work}
In the same vein, complex algorithms in this family
  \Dash those that quickly become too complex for a human to keep track of \Dash
  can be difficult to test without spending significant time
  actually \emph{implementing} the algorithm from scratch in a programming language.
This is a great barrier to some more mathematically-bent researchers in the field,
  given that they may have little to no experience in modeling mathematical systems as a program.
Even for those that \emph{are} comfortable with the practice,
  there are few tools that are readily available to assist in the work.
One that seems to stand out from the rest is Python's NetworkX~\autocite{hagberg:networkx},
  an extensive graph representation and manipulation library.
While these libraries assist in modeling the raw graph,
  they provide little to no assistance in creating, testing, or maintaining the algorithm itself.
They are simply not designed for such a relatively specific task.

This is not to say that there aren't many excellent frameworks in which
  to \emph{deploy} self-stabilizing algorithms.
For example, "hpc/libcircle" is \enquote{an API to provide an efficient distributed queue on a cluster}.
Using a blazingly fast algorithm described by~\autocite{lafon:balanceMQ},
  it is able to \enquote{quickly traverse and perform operations on a file tree
    which contains several hundred-million file nodes}.
This is a good tool to use, but is unfriendly as far as self-stabilization research is concerned.
The tool is written in and requires the use of C
  \Dash an \emph{excellent} choice of language in industries where speed is of the essence \Dash
  but it would seem unwise to create the algorithm itself
  (and test it interactively) with such a language.

On GitHub, "fintler/balancemq" provides another such API, this time written in Python,
  using an algorithm described by \citeauthor{lafon:balanceMQ}~\autocite{lafon:balanceMQ}.
Again, while claiming to be an API for self-stabilizing algorithms,
  this (albeit useful) project does not provide true tools to create and work with
  fault-tolerant systems in more general, simpler cases.
Nevertheless, these projects are excellent examples of the
  usefulness of these algorithms to the industry and
  the necessity of their existence and exploration.
\todo[cite]{How do I cite GitHub projects?
  I'm not sure who they're by, but they cite papers containing the algorithms used.}

\subsection{Current Work}
\label{sec:introduction:current-work}
Currently presented is a Python framework within which to
  create, define, and test these algorithms.
Complete with a (work in-progress) graphical interface which supports
  maintaining libraries of predicate and moves for later assembly,
  "ssa-tool" provides researchers with useful functionality
  to create and test algorithms in batch and by hand.

\paragraph{Necessity and Purpose}
While the tools above utilize self-stabilizing algorithms for specific applications,
  appealing themselves more to systems administrators than academics,
  "ssa-tool" aims to be general and robust enough to handle
  the wide variety of algorithms produced by scholarly research.
"ssa-tool" represents self-stabilizing algorithms and their components
  as closely as possible to the available mathematical definitions.
In this way, it aims to be a useful resource to quickly support published works
  and to provide the means to test algorithms \emph{before} publication,
  even to the point of creative construction.

\paragraph{Philosophy}
The tool is built upon a philosophy of accessibility;
  all core functionality is completely text-based.
By using YAML~\autocite{yaml:ref} as the basic database structure
  and defining the algorithm steps with short Python snippets,
  "ssa-tool" maintains a simple file structure that is easily
  understandable and editable by a human hand
  without sacrificing straightforward
  manipulation and execution by a computer.
This adherence to a lower level of operations and
  avoidance of higher-level functionality is deliberate;
  it is my experience that a sound engine, however bare it may be,
  is \emph{essential} to build a functional, more user-friendly interface.
By keeping the core system simple enough
  for a human to intuitively understand
  and using industry-standard technologies,
  it stands to reason that more feature-rich graphical interfaces
  may be laid atop this system
  by those more adept in developing them.
It also reduces the chance and general opportunity for bugs to arise,
  ensuring a stable engine that can be easily extended.

\paragraph{Potential}
While the allotted term-time for development is drawing to a close,
  feature-freeze has set in.
Considering the above philosophy, this is likely for the better.
However, there are many features that "ssa-tool" provides the backbone for:
\begin{itemize}
\providecommand\TaskRef[1]{\hfill\mbox{see~\autoref{task:#1}}}
\item The separation of engine from interface (even the rudimentary textual interface)
  allows for the painless of better interfaces.
  \TaskRef{interface}
\item The use of NetworkX~\autocite{hagberg:networkx} provides a possible
  interface to popular graph visualization applications (such as Gephi~\autocite{gephi}).
  \TaskRef{gephi-glue}
\item A complete log of changes is kept, feasibly allowing animation for
  output formats that support it (such as \fileformat{gexf}).\footnote{%
    NetworkX does not yet support animated graphs.
    Outputting to this format will take quite a bit of work
      unless progress is made by the NetworkX project.
    See~\autoref{task:anigexf}.}
  \TaskRef{anigexf}
\end{itemize}
See~\autoref{sec:further-work} for a complete list of these tasks.

\subsection{Moh' Features, Moh' Problems}
For those who wish to use it,
  "ssa-tool" also comes with a basic graphical interface (written with Tkinter~\autocite{tkinter})
  for the construction and manipulation of \enquote*{bundles} \Dash
  self-contained collections of algorithm descriptions
  with all necessary components (predicates\slash rules)
  needed to run them.\footnote{This structure is detailed in~\autoref{sec:bundle-descr-docum}.}
The use of this specific interface is documented in~\autoref{sec:interface-ssa}.

As with all graphical interfaces,
  there is some functionality of "ssa-tool" that it does not support.
\begin{itemize}
\providecommand\TaskRef[1]{\hfill\mbox{see~\autoref{task:#1}}}
\item It currently makes no provision for testing or running tests.
  \TaskRef{testing-interface}
\item While it does provide templates upon predicate\slash move creation,
  it does not provide any form of syntax highlighting during their creation
  as is common with most editors.
  \TaskRef{gui-syntax-highlighting}
\item When writing definitions for predicates\slash moves,
  there are no scrollbars and lines wrap, potentially confusing the user.
  \TaskRef{gui-definition-wrapping}
\item The prototype graph visualization tool
  (written with PyGame~\autocite{pygame})
  remains unintegrated into the manager.
  \TaskRef{gui-visualizer}
\end{itemize}
These tasks are left as goals for future versions to be considered for a truly public release.
The current interface, as it stands, is a \emph{prototype}; it is neither thoroughly tested
  nor fully representative of "ssa-tool"'s feature-set.
See~\autoref{task:test}.

%%% Local Variables:
%%% mode: latex
%%% TeX-master: "../smp.tex"
%%% TeX-PDF-mode: t
%%% reftex-cite-format: "\\autocite{%l}"
%%% TeX-command-default: "arara"
%%% TeX-engine: xetex
%%% End:
