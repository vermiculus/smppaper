\section{Introduction}
\label{sec:introduction}
\done\todo{Author-cite Dijsktra, Goddard}
In his seminal paper on the topic~\autocite{Dijkstra:1974:SSS:361179.361202},
  \citeauthor{Dijkstra:1974:SSS:361179.361202} introduced the concept of
  self-stabilizing algorithms.
He visualized the realization of these algorithms as taking advantage of
  a very specific, extensible scenario:
\begin{displayquote}[\citeay{Dijkstra:1974:SSS:361179.361202}]
  We consider a connected graph in which
    the majority of the possible edges are missing and
    a finite state machine is placed at each node;
    machines placed in directly connected nodes are called each other's neighbors.
  For each machine one or more so-called \enquote{privileges} are defined,
    \ie boolean functions of its own state and the states of its neighbors;
    when such a boolean function is true,
    we say that the privilege is \enquote{present}.
  In order to model the undefined speed ratios of the various machines,
    we introduce a central daemon \Dash
    its replacement by a distributed daemon falls outside the scope of this article \Dash
    that can \enquote{select} one of the privileges present.
  The machine enjoying the selected privilege will then make its \enquote{move};
    \ie it is brought into a new state that is a function of
    its old state and the states of its neighbors.
  If for such a machine more than one privilege is present,
    the new state may also depend on the privilege selected.
  After completion of the move, the daemon will select a new privilege.
\end{displayquote}
This canonical definition is inherently limited, however;
  each node $n$ only knows the state of its distance-1 neighbors $v \in \ClosedNeighborhood{n}$.
\citeauthor{gairing:distance-2} introduced an approach for
  simulating distance-2 information~\autocite{gairing:distance-2},
  which was later generalized to simulating distance-$k$
  information~\autocite{goddard:ssa--k-distance} by~\citeauthor{goddard:ssa--k-distance}.
  (The approach necessarily introduced ploynomial time and space complexities.)
\citeauthor{Dijkstra:1974:SSS:361179.361202}'s restrictive-yet-practical definition
  of a self-stabilizing algorithm
  was now \emph{far} less restrictive,
  since \citeauthor{goddard:ssa--k-distance} allowed \emph{any} distance-$k$ algorithm
  to be simulated in the original definition within a reasonable complexity increase.

Since then, many such algorithms have been put forward, unsurprisingly.
\todo[cite]{find some articles}
Given their innately parallel nature,
  self-stabilizing algorithms are \emph{incredibly} useful
  for modeling and solving real-world problems.

\subsection{Related Work}
\label{sec:introduction:related-work}
In the same vein, these algorithms can be incredibly hard to test
  \Dash let alone prove \Dash
  without spending significant time implementing the algorithm
  from scratch in a programming language.
For some programming languages,
  there exist extensive graph representation libraries (\eg NetworkX~\autocite{hagberg:networkx}).
While these libraries assist in modeling the graph itself,
  they provide little to no assistance in creating, testing, or maintaining the algorithm itself.

It seems that this is partly due to the lack of definition as to
  \emph{what} exactly a self-stabilizing algorithm is.

\begin{draftvspace}{5in}
  Expand on related work.
  There are a couple of other things available (see another reference~\autocite{lafon:libcircle}, for instance),
    but I've yet to do the necessary research to say \emph{what} exactly they do.
  All I know (and cared about) was that I wasn't \emph{duplicating} functionality.
\end{draftvspace}

\subsection{Current Work}
\label{sec:introduction:current-work}
This paper presents and defends such a library and toolset.
Complete with a graphical interface which supports
  maintaining libraries of \object{Predicate} and \object{Move} objects for later assembly,
  \texttt{ssa-tool} provides researchers with tools to test algorithms in batch and by hand.
\begin{draftvspace}{4in}
  Absolutely \emph{must} expand on this more.
  Go into what exactly happens and why it is necessary.
\end{draftvspace}

%%% Local Variables: 
%%% mode: latex
%%% TeX-master: "../smp.tex"
%%% TeX-PDF-mode: t 
%%% reftex-cite-format: "\\autocite{%l}" 
%%% TeX-command-default: "arara"
%%% End: 
