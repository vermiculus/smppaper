\section{Introduction}
\label{sec:introduction}

In his paper~\autocite{Dijkstra:1974:SSS:361179.361202}, Dijkstra introduced
  the concept of self-stabilizing algorithms by providing terminology and an example.
He visualized the realization of these algorithms as taking advantage of
  a very specific, extensible scenario:
\begin{displayquote}
  We consider a connected graph in which
    the majority of the possible edges are missing and
    a finite state machine is placed at each node;
    machines placed in directly connected nodes are called each other's neighbors.
  For each machine one or more so-called \enquote{privileges} are defined,
    \ie boolean functions of its own state and the states of its neighbors;
    when such a boolean function is true,
    we say that the privilege is \enquote{present.}
  In order to model the undefined speed ratios of the various machines,
    we introduce a central daemon \Dash
    its replacement by a distributed daemon falls outside the scope of this article \Dash
    that can \enquote{select} one of the privileges present.
  The machine enjoying the selected privilege will then make its \enquote{move};
    \ie it is brought into a new state that is a function of
    its old state and the states of its neighbors.
  If for such a machine more than one privilege is present,
    the new state may also depend on the privilege selected.
  After completion of the move, the daemon will select a new privilege.
\end{displayquote}
Paraphrasing, we consider a graph~$G$ with nodes $n_i \in G$ where
  every $n_i$ is given the properties of a \gls{nfa}.
Considering only one $n_i$ and its representative \gls{nfa} $M$ in some state $M_k$,
  the set of its privileges is understood as transitions
  from $M_k \to M_i$ where $M_i$ is another state in $M$.
  \todo[reword]{It's awkward phrasing as it is.  Perhaps I can bring in $\delta$?}
When such a transition exists on input,
  we call the privilege is \Term{present.}
The node is stable once $M_k$ is an accepting state,
  and the system has stabilized once every $n_i \in G$ is stable
  \emph{and} no privileges are present.

Self-stabilizing algorithms are incredibly useful for real-world problems.

%%% Local Variables: 
%%% mode: latex
%%% TeX-master: "../smp.tex"
%%% TeX-PDF-mode: t 
%%% End: 
